This file is an initial attempt to 
   understand various alternative 
   approaches to a constructive theory 
   of algebraic reductions.

   Sections: 

   Basics. Basic type and operator definitions. 

   Properties. Algebraic properties for equivalence 
      relations and semigroups. 

   EqualityTheory. Record structure to hold proofs for 
      equivalence relations, plus a few useful lemmas. 

   Semigroups. Record structures to hold all the bits of a semigroup. 

   ProductTheory. Property preservation theorems for 
      brel_product and bop_product operators. 

   Product_Semigroup. Uses results of ProductTheory to define 
      combinators constructing product semigroups. 
      This is just a simple example of how to build "combinators"

   ReductionProperties. Properties on unary operators capturing reductions. 
      NEW: split reduction_proofs record into two records: reduction_eqv_proofs and reduction_bop_proofs

   ReductionTheory. This is a "direct representation" of 
      reductions r using proof-carrying type {s : S & eq (r s) s = true}. 
      Binary operator must manipulate proof objects. 
      Not extraction friendly! 
      NEW in this section : exploration of isomorphism between "direct" and "indirect" representation. 
          This should *guide* out design of "rsemigroups" ... 
      NEW isomprhoism lemmas: 
          inj_homomorphism
          proj1_homomorphism
          inj_proj1_is_id
          proj1_inj_is_id_on_image_of_r
          congruence_iso 
          commutative_iso_1
          commutative_iso_2
          sanity 
          commutative_iso
          associative_iso

   Reduced_Semigroup_Direct. Uses results of ReductionTheory to define 
      combinators constructing reduced semigroup and products 
      of reduced semigroups (all using "direct representation"). 
      NOT extraction friendly! 

   Reduced_Semigroup. Inspired by the "isomorphism theorems" of 
      Section ReductionTheory, define "extraction friendly" 
      semigroups that are isomomprhic to the direct representation. 

   Rsemigroup. NEW (well, completely redone). 
       There is a problem with the approach of the last section. 
       We have lost the reduction r --- but this is needed for 
       injecting elements of S into the reduced subset of S. 
       This section introduces a new structure, rsemigroup, that 
       remembers r. 

   Product_Rsemigroup.  This is simply a rewrite 
     of section Product_Semigroup using rsemigroups. 

   SimpleExample. A simple sanity check that everything works.  

   Reduce_Annihilators. NEW. A more interesting example! 
      This section contains theory and defintion of a new 
      generic reduction that reduces annihilators in pairs. 
      NEEDS TO BE FINISHED 

   Reduced_Rsemigroup. NEW. Can we reduce rsemigroups?    
      Just some NOTES on how hard this is going to be ... 




THE BIG QUESTION: CAN WE REDUCE RSEMIGROUPS? 

 Some notes on this questions: 

Suppose we have a reduction r1 over bop b0 : 

   b1 x y = r1(b0 (r1 x) (r1 y)) 
   r1_idem  : r1 (r1 x) = r1 x 
   r1_left  : r1 (b0 (r1 s1) s2) = r1 (b0 s1 s2)
   r1_right : r1 (b0 s1 (r1 s2)) = r1 (b0 s1 s2)

Now, suppose we want to define reduction r2 over b1.  New need

    b2 x y = r2(b1 (r2 x) (r2 y)) 
           = r2(r1(b0 (r1 (r2 x)) (r1 (r2 y))))
           = (r2 . r1)(b0 ((r1 . r2) x) ((r1 . r2) y))   ( the . is function composition) 

    r2_idem : r2 (r2 x) = r2 x 

    r2_left  : r2 (b1 (r2 s1) s2) = r2 (b1 s1 s2)
           <-> r2 (r1 (b0 (r1 (r2 s1)) (r1 s2)) = r2 (r1(b0 (r1 s1) (r1 s2)) )
           <-> (r2 . r1) (b0 ((r1. r2) s1) (r1 s2)) = (r2 . r1) (b0 (r1 s1) (r1 s2)) )
    r2_right : r2 (b1 s1 (r2 s2)) = r2 (b1 s1 s2)

Note something rather unexpected here:  we now need to reason about all of these
functions: 
   1) r1 
   3) r2 . r1  
   3) r1 . r2
and their interactions with b0.  This may be very difficult in general .... 

