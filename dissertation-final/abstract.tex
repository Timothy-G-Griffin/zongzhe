\newpage
{\Huge \bf Abstract}
\vspace{24pt} 

%This topic is originated from the expansion of a routing problem that was not specifically reasoned in the L11 class. We classify this class of problems as a specific type of problem, reduction. Although we discussed some of the properties of reduction in L11, however, the construction of the reduction for what we need in the final construction of path problem will conflict with the definition of reduction itself. Therefore, it shows that our definition of the reduction does not resolve our actual problem.
%
%When we traced the source back, we found the definition of reduction from a paper by Wongseelashote in 1979. Although Wongseelashote very skilfully proposed the concept of reduction when discussing path problems, it is regrettable that this paper did not provide detailed structural proof of reduction for its properties.
%
%Therefore, our project is comprised of the following three parts. First of all, the classical reduction is expressed, and we are trying to reason the properties of reduction itself. Next, we will try to represent/define reduction in another way, not only to facilitate the implementation, but also to decrease the limitation of the reduction definition to practical problems. That generalised reduction in good representation could help us defining the reduction that we need in the problem. After that, we will define a kind of reduction according to our requirements, and use this kind of reduction to construct numerous reduction for realistic examples. Finally, we will use these practical examples as a combination to resolve the path problem that was mentioned in L11 lecture.

This dissertation presents an in-depth exploration of algebraic reductions first introduced by Wongseelashote in 1979. All of our mathematical work was done using the Coq assistant prover. However, the dissertation presents our results in traditional informal mathematics. Our Coq implementation will be integrated into Dr Griffin's Coq code for Combinators of Algebraic Systems (CAS).

\newpage
\vspace*{\fill}
