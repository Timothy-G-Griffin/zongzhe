%% 
%% ACS project dissertation template. 
%% 
%% Currently designed for printing two-sided, but if you prefer to 
%% print single-sided just remove ",twoside,openright" from the 
%% \documentclass[] line below. 
%%
%%
%%   SMH, May 2010. 


\documentclass[a4paper,12pt,twoside,openright]{report}


%%
%% EDIT THE BELOW TO CUSTOMIZE
%%

\def\authorname{Zongzhe Yuan\xspace}
\def\authorcollege{Christ's College\xspace}
\def\authoremail{zy272@cl.cam.ac.uk}
\def\dissertationtitle{Extending CAS with Algebraic Reductions}
\def\wordcount{14,235}


\usepackage{epsfig,graphicx,parskip,setspace,tabularx,xspace} 
\usepackage{a4wide,parskip,times}
\usepackage{amsmath}
\usepackage{nicefrac}
\usepackage{ amssymb }
\usepackage{mleftright}
\usepackage{listings}

%% START OF DOCUMENT
\begin{document}


%% FRONTMATTER (TITLE PAGE, DECLARATION, ABSTRACT, ETC) 
\pagestyle{empty}
\singlespacing
\input{titlepage}
\onehalfspacing
\newpage
{\Huge \bf Declaration}

\vspace{24pt} 

I \authorname of \authorcollege, being a candidate for the M.Phil in
Advanced Computer Science, hereby declare that this report and the
work described in it are my own work, unaided except as may be
specified below, and that the report does not contain material that
has already been used to any substantial extent for a comparable
purpose.

The project and dissertation are finished under the guidance of my supervisor Dr Griffin.

The appendix contains the additional verbose elements of the Coq proof content, which is not included into the main thesis, and is only used to provide a better understanding for the readers.

\vspace{24pt}
Total word count: \wordcount \\
(excluding the words from appendix)

\vspace{60pt}
\textbf{Signed}: 

\vspace{12pt}
\textbf{Date}:


\vfill

This dissertation is copyright \copyright 2018 \authorname. 
\\
All trademarks used in this dissertation are hereby acknowledged.



\newpage
\vspace*{\fill}

\singlespacing
\newpage
{\Huge \bf Abstract}
\vspace{24pt} 

%This topic is originated from the expansion of a routing problem that was not specifically reasoned in the L11 class. We classify this class of problems as a specific type of problem, reduction. Although we discussed some of the properties of reduction in L11, however, the construction of the reduction for what we need in the final construction of path problem will conflict with the definition of reduction itself. Therefore, it shows that our definition of the reduction does not resolve our actual problem.
%
%When we traced the source back, we found the definition of reduction from a paper by Wongseelashote in 1979. Although Wongseelashote very skilfully proposed the concept of reduction when discussing path problems, it is regrettable that this paper did not provide detailed structural proof of reduction for its properties.
%
%Therefore, our project is comprised of the following three parts. First of all, the classical reduction is expressed, and we are trying to reason the properties of reduction itself. Next, we will try to represent/define reduction in another way, not only to facilitate the implementation, but also to decrease the limitation of the reduction definition to practical problems. That generalised reduction in good representation could help us defining the reduction that we need in the problem. After that, we will define a kind of reduction according to our requirements, and use this kind of reduction to construct numerous reduction for realistic examples. Finally, we will use these practical examples as a combination to resolve the path problem that was mentioned in L11 lecture.

This dissertation presents an in-depth exploration of algebraic reductions first introduced by Wongseelashote in 1979. All of our mathematical work was done using the Coq assistant prover. However, the dissertation presents our results in traditional informal mathematics. Our Coq implementation will be integrated into Dr Griffin's Coq code for Combinators of Algebraic Systems (CAS).

\newpage
\vspace*{\fill}


\pagenumbering{roman}
\setcounter{page}{0}
\pagestyle{plain}
\tableofcontents
\listoffigures
\listoftables

\onehalfspacing

%% START OF MAIN TEXT 

\chapter{Introduction}
\pagenumbering{arabic} 
\setcounter{page}{1} 

%This is the introduction where you should introduce your work.  In
%general the thing to aim for here is to describe a little bit of the
%context for your work --- why did you do it (motivation), what was the
%hoped-for outcome (aims) --- as well as trying to give a brief
%overview of what you actually did.
%
%It's often useful to bring forward some ``highlights'' into 
%this chapter (e.g.\ some particularly compelling results, or 
%a particularly interesting finding). 
%
%It's also traditional to give an outline of the rest of the
%document, although without care this can appear formulaic 
%and tedious. Your call. 

\section{Introduction to the Path Problem and Algebraic Solution}


The path problem has always fascinated mathematicians and computer scientists. 
At the very beginning, programmer and scientists designed algorithms to solve each of the path problem. 
The most famous path problem is the shortest distance problem and there are several well-known algorithm that can solve such a problem: Dijkstra's algorithm, Bellman–Ford algorithm, A* search algorithm and Floyd–Warshall algorithm. People use different primitive metrics and various complicated algorithms to solve different path problems.\\\\
However, such an approach has its obvious shortcomings. Even at some point, designing a new algorithm can "steal" the ideas of the original algorithm, people must design a completely new independent algorithm in the face of each new problem (new metric), and this makes it difficult to have a generic (or framework) approach to solve this type of problem.  Even if the path problem has minor changes to the problem, it is difficult for people to solve the new problem by slightly modifying existing algorithms. \\
Hence, lots of predecessors have found the algebraic approaches to work around this kind of problem. 
Using the knowledge of abstract algebra, people find that the routing problem (path problem) can be represent using a data structure called "semiring" $(S,\oplus,\otimes,\bar{0},\bar{1})$\cite{carre_algebra_1971,WONGSEELASHOTE197955,dynerowicz_forwarding_2013,mohri_semiring_2002,gurney_lexicographic_2007}. For example, the popular "shortest path problem" can be represented as $(S, min,+,\infty,0)$\cite{mohri_semiring_2002} and the "maximal capacity path problem" can be represented as $(S, max,min, 0, \infty)$. \\\\
For each path problem that represented as a semiring, we can construct a corresponding matrix semiring that represent the concrete problem (the edges and the distances for the shortest path problem for example). Then using matrix multiplication and stability of the closure (the semiring), we can solve the real problem of each concrete path problem. However, the simple matrix approach can only solve the "trivial" path problem. Those complicated problem, for example the widest shortest path problem that constructed from the shortest path problem semiring and the widest path problem (maximal capacity path problem) semiring using lexicograhpic product, can't be solved by this "traditional" theory approach. Some times even the method can find an optimal solution, there is no guarantee to find all optimal solutions using the classical method.\\\\
Therefore, people have found a non-classical theory of algebraic path finding method, so that algebras that violated the distribution law can be accepted. This non-classical theory can handle the problem the simple classical theory can't handle, such as the problem that can't be solved by Dijkstra or Bellman-Ford.
This kind of method is dedicated to finding the local optimal solution at first, and then the local optimal solution is exactly the same as the global optimal solution by some verification or some addition restriction on the computation. For example, the famous protocol, the routing information protocol which is based on distributed Bellman-Ford algorithms is one of the non-traditional theory.\\\\
However, even so, RIP will also have a series of problems. For example, when a node does not have edges connected to it, the RIP matrix calculation (without pre-setting the maximum number of calculation steps) will continue infinitely. Even if the maximum number of calculation steps is set in advance, RIP still has some deficiencies in efficiency. \\\\
So while we use another protocol BGP (Border Gateway Protocol), we add an annihilator to the entire complex semiring.  At the same time, we found that when we represent the path, we may have a loop path (a node in the path has been passed more than once). So we need our problem set to change from the original path to elementary path (A path p is elementary if no node is repeated). In this process, we performed two operations $S \rightarrow S$, and here we call it reduction in general, which comes to the main problem of our project.\\\\

\section{Reduction}
Algebraic reduction, introduced by Ahnont Wongseelashote in 1977\cite{WONGSEELASHOTE197955} is an unary operator for a given set of problem, $reduce : S \longrightarrow S$. It has several properties, satisfying $reduce(\emptyset) = \emptyset$, $\forall A \in S, reduce(reduce(A)) = reduce(A)$ (which is called idempotent property) and $\oplus : S \times S \rightarrow S, \forall A,B \in S, reduce(reduce(A)\oplus B) = reduce(A\oplus B) = reduce(A\oplus reduce(B))$ (which is left and right invariant property) and this paper will discuss these properties in the later section.\\ 
It is hard to specify the reduced problem set in the most programming language. However, in the world of logic and those programming language that can be used to prove properties, programmer can represent the reduced problem set as $\{x | x \in S, reduce(x) = x\}$ which is also a subset of the original problem set. The idea to represent the reduced set explicitly is to form a pair $<x,Pr(x)>$ where $x \in S$ is the element in the set and $Pr(x) : r(x) = x$ is the proof that the element is in the subset (the element will not change after the reduced function, otherwise it will be reduced). \\
The first example of the reduction is $id$ which maps all the stuff to itself. Another example is the min-set where $min_\leq(x) = \{x \in S | \forall y \in S, \neg(y < x)\}$. Regarding to $\mathcal{P}(S)$ it works well with $\cup$ that the min-set contain all the elements and remove the element that is non-trivial set. And this reduction is used in the construction of elementary path. 


\section{Motivation}
Wongseelashote defined the reduction operator in his paper\cite{WONGSEELASHOTE197955}. However the definition there is not constructive and it is the traditional reduction.
Gurney and L11 claimed reductions could be used for "non-traditional" reductions such as elementary paths and combining elementary paths with lexicographic product \cite{gurney_construction_2010}. 
However, Gurney and L11 never worked out the details, and the reduction there is still not constructive. \\\\

It is worth mentioning that, the algebraic approaches that using the matrix as the computation rely on the property of the operators a lot, for example, the left and right distributivity of the semiring.
However, most of the cases mentioned above are aimed at some simple problems, or the ideal situations. 
In reality, we need to face the problem that, in the most time for the complex path problem (especially for lexicographic product). 
The CAS system can derive most of the properties for the new semirings from the original "simple" simirings. 
However, sometimes the problem set is not the original problem set the provided to us, but the subset of it. As I mentioned before, as the problem that consist of the lexicographic product of the shortest path problem semiring and the maximum capacity semiring, there exists some path that have 0 capacity which shouldn't be concerned in the solution of the problem. This kind of reduction can be represent as a unary operation on the original problem set in our paper.\\\\
Another example, when we are doing path problems, for the most time the path that has negative value or have infinite value is not quite interesting. However, the operation we defined there is on the whole families of object. When we are defining some data structure, like semiring, we want to know that the properties (like commutative, selective and etc..) of the proper subset (the set of objects after reduction) will hold or not, or for some cases we can't do further calculation.\\\\
As the method mentioned previously, the algebraic approach to the path problem is depend on the properties of semirings. If one, or some of the properties don't hold for the semiring, the algebraic approach can't guarantee to find the optimal solution (depends on the semiring structure and the operations). Thus, after we applying those reductions on the original problem set, there is no guarantee that the original properties will still hold for the new subset of problem set, and it comes to our point. The existing CAS system doesn't have the functionality to derive and prove the properties for the reduced problem. Hence, I want to figure out the relationship between the reduction and those properties for those operators on the problem set.\\\\
Furthermore, in most programming languages, it is extremely difficult to express the reduction properly, on contrast, we can represent the reduction as a proof or proposition in our proof world. The second goal to the project is to figure out the friendly-extraction to those reduction. 



\chapter{Background} 

%A more extensive coverage of what's required to understand your 
%work. In general you should assume the reader has a good undergraduate 
%degree in computer science, but is not necessarily an expert in 
%the particular area you've been working on. Hence this chapter 
%may need to summarize some ``text book'' material. 
%
%This is not something you'd normally require in an academic paper, 
%and it may not be appropriate for your particular circumstances. 
%Indeed, in some cases it's possible to cover all of the ``background'' 
%material either in the introduction or at appropriate places in 
%the rest of the dissertation. 

This chapter will introduce several basic concept, and some related ideas that will help us to understand the main idea of the project.

\section{Basic Definition}
In the world of logic, we need to define several basic concepts before we are really getting started.\\
We can simply understand that a Type is a collection of terms which have the same "properties". \\
Under a Type $S$, we can define a binary relationship  (in most cases we will define the binary relationship of equality $=_S$ at first).
In mathematics, A binary relation $R$ in an arbitrary Type $S$ (here I restrict the relationship to be inside a single Type, or say the element from the same type) is a collection of ordered pairs of elements of set (type) $S$, which is a subset of the Cartesian product $S \times S$. \\
In order to link the mathematical concept with the proposition in logic, I provides each relation a representation (hold or not) as a boolean value, which is regarded as a property in $Coq$. \\
$\Lambda S. brel : S \rightarrow S \rightarrow bool$. \\
Next we should define the operators that exist on a given Type:\\
binary operation $: \Lambda S.binary\_op :S \rightarrow S \rightarrow S$\\
unary operation $: \Lambda S.unary\_op:S \rightarrow S$. \\
\subsection{Properties for the Relationship}
For our definition of binary relationship, we need to focus on a few properties related to it.\\
Here we have four properties for a binary relationship for type $S$. As it is mentioned in the previous section, in order to define the proposition in the logic proof, the binary relationship will return a boolean value indicating that the relationship is holding or not, here we provide a arbitrary type variable $S$.\\
Then for a given relationship $=_S : S \rightarrow S \rightarrow bool$, we should define:\\
reflexivity: $\forall a \in S, a =_S a \equiv true$,\\
symmetric : $\forall a, b \in S, a =_S b \equiv b =_S a$, \\
transitivity: $\forall a,b,c \in S, a =_S b \equiv true \wedge b =_S c \equiv true \rightarrow a =_S c \equiv true$, \\
congruence : $\forall a,b,c,d \in S, a =_S b \equiv true \rightarrow c =_S d \equiv true \rightarrow a =_S c \equiv b = _S d$
\subsection{Properties for the Operator}
Next we need to discuss some of the properties of the unary operator and the binary operator that we have defined.
\subsubsection{Properties for Unary Operator}
For the unary operator we have defined, we want it to have the following properties, or we will analyse the unary operator we have defined from the following perspectives. \\
In fact, our unary operator is used to represent reduction, so our discussion of the properties of the unary operator is actually a discussion of some of the properties of reduction.\\
There are six properties that we need to be discussed. Three of them are of the same properties as the reduction mentioned in the previous section, here we give a arbitrary type variable $S$, a binary relationship in $S$ : $=_S$ and our unary operation $u_S$.\\
congruence : $\forall a,b \in S, a =_S b \rightarrow u_S(a) =_S u_S(b)$,\\
idempotent : $\forall a \in S, u_S(u_S(a)) =_S u_S(a)$,\\
left\_invariant: $\Lambda \oplus _S: S \rightarrow S \rightarrow S, \forall a,b \in S, u_S (a) \oplus_S b =_S a \oplus_S b$,\\
right\_invariant: $\Lambda \oplus _S: S \rightarrow S \rightarrow S, \forall a,b \in S, a \oplus_S u_S(b) =_S a \oplus_S b$,\\
preserve\_id : $\forall i \in S$, if $i$ is the identity for type $S$, equality $=_S$ and a binary operator $\oplus_S$, ($\forall a \in S, a \oplus_S i =_S a =_S i \oplus_S a$) then $u_S (i) =_S i$,\\
preserve\_annihilator : $\forall a \in S$, if $i$ is the annihilator for type $S$, equality $=_S$ and a binary operator $\oplus_S$, ($\forall b \in S, a \oplus_S b=_S a =_S b \oplus_S a$) then $u_S (b) =_S b$.



\subsubsection{Properties for Binary Operator}
Also for the binary operator we have defined, we hope that we will focus on the following seven properties of it.
For a given type variable $S$, the binary relationship $=_S$ and our pre-defined binary operator $\oplus_S: S\times S \rightarrow S$, we have:\\
congruence: $\forall s_1,s_2,t_1,t_2 \in S, s_1 \equiv t_1 \wedge s_2 \equiv t_2 \rightarrow s_1 \oplus_S s_2 \equiv t_1 \oplus_S t_2$,\\
idempotent : $\forall a \in S, a \oplus_S a =_S a$,\\
associativity: $\forall a,b,c \in S, a \oplus_S (b \oplus_S c) \equiv (a \oplus_S b) \oplus_S c$,\\
commutativity: $\forall a,b\in S, a \oplus_S b = b \oplus_S a$,\\
selectivity: $\forall a,b\in S, a \oplus_S b \in \{a,b\}$,\\
hasId: $\exists \bar0 \in S, \forall a \in S, a \oplus_S \bar0 = a = \bar0 \oplus_S a$,which means $\bar0$ is the identity for type $S$, equality $=_S$ and the binary operator $\oplus_S$,\\
hasAnn: $\exists \bar1 \in S, \forall a \in S, a \oplus_S \bar1 = \bar1 = \bar1 \oplus_S a$, which means $\bar1$ is the identity for type $S$, equality $=_S$ and the binary operator $\oplus_S$,\\
In this paper, we will mostly concentrate on those properties that are holding or not in the reduced set of problem. 


\section{Direct Product and its Properties}
Here we need to mention Product Type and some properties about Product Type. \\
CAS will use pair to combine two semirings (or just semigroups) to form a new data structure. 
Therefore we must discuss the relationship between the properties of the single semirings and the properties they have been combined with. \\
First, as with a given Type and its binary relationship, we need to define a binary relationship to describe the relationship between product type terms.\\
$\Lambda S, T, =_s, =_t. brel\_product : S * T \rightarrow S * T \rightarrow bool$, \\
$:= \lambda (s_1,t_1) (s_2, t_2). s_1 =_s s_2 \wedge t_1 =_t t_2$, (we can write this binary relationship as $=_S \times =_T$).\\
At the same time, we should give an expansion on existing unary operators and binary operators for product.\\
For two given type $S$ and $T$, unary operators $u_S$ and $u_T$, binary operators $\oplus_S$ and $\oplus_T$:\\
uop\_product: $S*T \rightarrow S*T := \lambda (s,t), (u_S (s),u_T(t))$,\\
bop\_product: $S*T \rightarrow S*T \rightarrow S*T := \lambda (s_1,t_1)(s_2,t_2), (s_1 \oplus_S s_2,t_1 \oplus_T t_2)$, (we can write this new binary operator as $(\oplus_S,\oplus_T)$).
\subsection{Properties for the Relationship}
In fact, we also focus on whether or not some of the properties of the binary relationship previously defined apply to the product type. Although the proof of this part is not the key to our project. 
Fortunately, the CAS system has systematically proved the product type properties, including binary relationship and binary operator. \\
For two given type $S$ and $T$ and their corresponding binary relationship $=_S$ and $=_T$,\\
If both $=_S$ and $=_T$ are reflexivity, then $=_S \times =_T$ is reflexivity,\\
If both $=_S$ and $=_T$ are symmetric, then $=_S \times =_T$ is symmetric,\\
If both $=_S$ and $=_T$ are transitivity, then $=_S \times =_T$ is transitivity,\\
If both $=_S$ and $=_T$ are congruence, then $=_S \times =_T$ is congruence.
\subsection{Properties for the Operator}
As mentioned above, fortunately the CAS system has given us a systematic proof of the binary operator of the product type.\\
For two given type $S$ and $T$, binary operators $\oplus_S$ and $\oplus_T$:\\
If both $\oplus_S$ and $\oplus_T$ are congruence, then $(\oplus_S,\oplus_T)$ is congruence,\\
If both $\oplus_S$ and $\oplus_T$ are commutativity, then $(\oplus_S,\oplus_T)$ is commutativity,\\
If both $\oplus_S$ and $\oplus_T$ are associativity, then $(\oplus_S,\oplus_T)$ is associativity,\\
If both $\oplus_S$ and $\oplus_T$ have Id $\bar0_S$ and $\bar0_T$, then $(\oplus_S,\oplus_T)$ has Id $(\bar0_S,\bar0_T)$,\\
If both $\oplus_S$ and $\oplus_T$ have Ann $\bar1_S$ and $\bar1_T$, then $(\oplus_S,\oplus_T)$ has Ann $(\bar1_S,\bar1_T)$.\\
It is worth mentioning that although both $\oplus_S$ and $\oplus_T$ may be selective, this does not lead to the conclusion that $(\oplus_S,\oplus_T)$ is a selective.
The explanation of selective will be mentioned later in the lexicographic product section.

\section{Semiring and its Properties}
In abstract algebra, a semiring is a data structure $(S,\oplus,\otimes,\bar0,\bar1)$ where $S$ is a set (Type) and $\oplus,\otimes$ are two operators $:S\times S \rightarrow S$.\\\\
$(S,\oplus)$ is a commutative semigroup (has associative property) and $(S,\otimes)$ is a semigroup:\\ 
$\forall a,b,c \in S, a \oplus (b\oplus c) = (a \oplus b) \oplus c$, $a \oplus b = b \oplus a$,\\
$\forall a,b,c \in S, a \otimes (b\otimes c) = (a \otimes b) \otimes c$.\\\\ 
$\oplus,\otimes$ are also left and right distributive on $S$ : \\
$\forall a,b,c \in S: a \otimes(b \oplus c) = (a \otimes b) \oplus (a \otimes c)$, \\ 
$\forall a,b,c \in S:(a \oplus b) \otimes c  = (a \otimes c) \oplus (b \otimes c)$. \\\\
$\bar0$ is the identity of $\oplus$ and  $\bar1$ is the identity of $\otimes$: \\
$\forall a \in S, a \oplus \bar{0} = a = \bar{0} \oplus a$,\\
$\forall a \in S, a \otimes \bar{1} = a = \bar{1} \otimes a$. \\\\
Finally, $\bar0$ is the  annihilator of $\otimes$: \\
$\forall a \in S, a \otimes \bar{0} = \bar{0} = \bar{0} \otimes a$.\\
Some definition will include that $\bar1$ is the annihilator of $\oplus$: \\
$\forall a \in S, a \oplus \bar{1} = \bar{1} = \bar{1} \oplus a$.


\section{Matrix Semiring and Stability}
For each semiring $(S,\oplus,\otimes,\bar0,\bar1)$ (that represent the rule of a path problem), we can define a matrix semiring $(M_n(S),\oplus,\otimes,\bar{J},\bar{I})$ to represent the concrete path problem.\\
$M_n(S)$ is a $n\times n$ matrices over $S$,\\
$(A \oplus B)(i,j) = A(i,j)\oplus B(i,j)$,\\
$(A \otimes B)(i,j) = \bigoplus_{1\leq q \leq n}A(i,q)\otimes B(q,j)$,\\
$\bar{J}(i,j) = \bar0$,\\
$\bar{I}(i,j)=\left\{
\begin{array}{rcl}
\bar1       &      & {i = j}\\
\bar0     &      & {otherwise}
\end{array} \right.$\\
So here we can easily use this matrix to encode any specific path problem and use matrix multiplication to calculate our answer. \\
For a graph $G = (V,E)$ a directed graph and a weight function $w \in E \rightarrow S$, 
we can define the weight of a path $p = i_1,i_2, \dots i_k$ is $w(p) = w (i_1,i_2) \otimes w(i_2,i_3)\otimes \dots \otimes w(i_{k-1},i_k)$, while the empty path is given the weight of $\bar1$.
Then we can define the adjacency matrix $A$ as 
$A(i,j)=\left\{
\begin{array}{rcl}
w(i,j)      &      & {(i,j)\in E}\\
\bar0     &      & {otherwise}
\end{array} \right.$\\
And our problem will be represent as $A^*(i,j) = \bigoplus_{p \in \pi (i,j)}w(p)$.\\\\
However, our calculations depend on the properties of the semiring.\\
$a\in S$,we define the powers $a^k$ as, $a^0 = \bar{1}$ and $a^{k+1} = a \otimes a^k$.\\
$a\in S$,we define the closure $a^*$ as, $a^{(k)} = a^0 \oplus a^1 \dots  a^k$ and $a^* = a^0 \oplus a^1 \dots  a^k \oplus \dots$.\\
Here we say, if there exists a $q$ such that $a^{q} = a^{(q+1)}$, then $a$ is q-stable, which means $a^{q} = a^*$.\\
And if we know that $S$ is 0-stable, then $M_n(S)$ is $n-1$-stable (because we can ignore paths with loops, and that is our reduction!). 
This allows us to actually calculate at most $n-1$ steps when calculating path problems.\\
Therefore we can define the power and closure on the matrix semiring,\\
$A\in M_n(S)$,we define the powers $A^k$ as, $A^0 = \bar{I}$ and $A^{k+1} = A \otimes A^k$.\\
$A\in M_n(S)$,we define the closure $A^*$ as, $A^{(k)} = A^0 \oplus A^1 \dots  A^k$ and $A^* = A^0 \oplus A^1 \dots  A^k \oplus \dots$.\\\\
Hence we have $\pi(i,j)$ which is the set of paths from $i$ to $j$, then $\pi^k(i,j)$ will be the set of paths from $i$ to $j$ with exactly $k$ arcs, then $\pi^{(k)}(i,j)$ will be the set of paths from $i$ to $j$ with at most $k$ arcs.\\
Then we have \\
$A^k(i,j) = \bigoplus_{p \in \pi^k (i,j)}w(p)$, \\
$A^{(k)}(i,j) = \bigoplus_{p \in \pi^{(k)} (i,j)}w(p)$ and\\
$A^*(i,j) = \bigoplus_{p \in \pi^* (i,j)}w(p)$.\\
It is worth mentioning that $A^*(i,j)$ may not be well defined, but if $M_n(S)$ is $k$ stable, then for $A\in M_n(S)$, $A^*(i,j) = A^{(k)}(i,j)$.

\section{Distributivity and Lexicographic Product}
It is worth mentioning that the properties of distributivity (left and right) play an important role in the computation of routing problem.
We've defined $M_n(S)$ the matrix and its related semiring on the given semiring $(S,\oplus,\otimes,\bar0,\bar1)$.
However we still need to check the properties of $M_n(S)$, to make sure that it is exactly a semiring.
Consider about the distributivity properties, $M_n(S)$ is left/right distributive if $S$ has the distributivity properties.\\
But some times when we are defining some complex routing problem, such as the widest-shortest path problem, which is constructed by shortest path problem semiring and widest path problem semiring by lexicographic product, it is not guarantee that the new data structure still have the properties of distributivity.\\\\
Suppose $(S,\oplus_S)$ is a commutative and selective semigroup and $(T,\oplus_t)$ is a semigroup, then the lexicographic product of two semigroups $(S,\oplus_S) \bar{\times} (T,\oplus_T) = (S\times T, \oplus_{\bar{\times}})$ where\\
$(s_1,t_1) \oplus_{\bar{\times}} (s_2,t_2)=\left\{
\begin{array}{rcl}
(s_1\oplus_S s_2,t_1\oplus_T t_2)      &      & { s_1 = s_1 \oplus_S s_2 = s_2}\\
(s_1\oplus_S s_2,t_1)       &      & {s_1 = s_1 \oplus_S s_2 \neq s_2}\\
(s_1\oplus_S s_2,t_2)       &      & {s_1 \neq s_1 \oplus_S s_2 = s_2}
\end{array} \right.$\\\\

As mentioned above, we used the lexicographic product when constructing the widest shortest path problem, and used the elementary path reduction. We also added a distinct annihilator in order to prevent excessive useless calculations. \\
These constructs make it difficult to speculate about the properties of our semiring (or new data structure), which is what our project is devoted to research that the relationship between reduction and semiring properties.

\section{Global optimality VS Left/Right local optimality}
On the other hand, we also need to discuss how to make our approach embrace the algebras that violate distributivity.\\
The global optimality: $A^*(i,j) = \bigoplus_{p \in \pi (i,j)}w(p)$,\\
The left local optimality which is the distributed Bellman-Ford algorithm: $L = (A\otimes L) \oplus \bar{I}$ which is  $L(i,j) = \bigoplus_{q \in V} A(i,q)\otimes L(q,j)$,\\
The right local optimality which is the Dijkstra's algorithm: $R = (R\otimes A) \oplus \bar{I}$ which is  $L(i,j) = \bigoplus_{q \in V} R(i,q)\otimes A(q,j)$,\\
note that with distributivity, $M_n(S)$ is a semiring and the three optimality problems are essentially the same, the local optimal solutions are global solutions :$A^* = L = R$,\\
However with out distributivity, those three solutions may all exists but all distinct, and that comes to the part of our project -- discussing about the relationship between reductions and semirings.


\section{Combinator for Algebraic System}
Combinator for Algebraic system (CAS)\cite{griffin_metarouting_2005} is introduced in $L11$. It is a language to design algebraic systems, in which many algebraic properties are automatically received and people can combine different operators to obtain a new semiring\cite{griffin_metarouting_2005}. We can also generalize a more complex path problem (in other words, we can abstract a more complex path problem with this new semiring, such as the lexicographic products \cite{gurney_lexicographic_2007}). CAS can easily return the properties of those combined-operation semirings and it is already defined in Coq\cite{Coq} (mentioned in $L11$ by Dr Timothy Griffin).


\chapter{Design and Implementation} 

%This chapter may be called something else\ldots but in general 
%the idea is that you have one (or a few) ``meat'' chapters which
%describe the work you did in technical detail. 

\section{Direct (Traditional) Reduction and Reduction Theory}
\subsection{Basic Definition}
Initially, First of all, I first made a detailed analysis (proof) of the reduction (we call it traditional reduction) mentioned by Wongseelashote in his article \cite{WONGSEELASHOTE197955}.\\
For a given type $S$, a binary relationship $=_S$ on S, a unary operator and a binary operator $u_S$ and $\oplus_S$, we define the traditional reduction as following:\\
$Pr (x : S) :=   u_S (x) =_S x$ which is the proof that, for a given element $x$ of type $S$, $x$ is in the reduced set.\\
$red\_Type := \{x : S \& Pr(x)\}$, which is a set that all the elements in the set are pairs of the element in $S$ itself together with the proof that the element is in the reduced set.\\
$red\_eq : red\_Type \rightarrow red\_Type \rightarrow bool 
:= \lambda p_1 p_2, projT_1 (p_1) =_S projT_1 (p_2)$. \\
Here $projT_1/projT_2$ are two 
auxiliary	function that map the element/the proof part from the reduced set.\\
It is worth mentioning that since the definition follows Wongseelashote in his article \cite{WONGSEELASHOTE197955}, our unary operator(reduction) needs congruence and idempotent properties.
And when we discuss the relation and properties of the unary operator (reduciton) and the binary operator, we need to have two properties left\_invariant and right\_invariant.
\\
With these definitions we can discuss the nature of the binary relationship of traditional reduction.
\subsection{Properties for the Relationship}
For our given type $S$ and binary relationship $=_S$, we assume that $=_S$ on $S$ has such properties:
\begin{itemize}
\item reflexive 
\item symmetric 
\item transitive 
\item congruence 
\end{itemize}
It is easy to prove that $red_eq$ on $red\_Type$ has those four properties:
\begin{itemize}
\item Lemma red\_ref : reflexive red\_Type red\_eq.
\item Lemma red\_sym : symmetric red\_Type red\_eq.
\item Lemma red\_cong : congruence red\_Type red\_eq red\_eq.  
\item Lemma red\_trans : transitive red\_Type red\_eq.  
\end{itemize}

\subsection{Properties for the Binary Operator}
Before starting our proof, we need to prove an auxiliary Lemma.\\
Lemma Pr\_br : $\forall (p1 p2 : red\_Type), Pr (u_S(projT1 (p1) \oplus_S projT1 (p2))).$\\
The significance of this Lemma is that for any two elements that exist in a reduced set, we take out its element separately and execute a binary operation, and finally do a reduction process, and the result is still in the reduced set.\\\\
Because our binary operator $\oplus_S$ is working on type $S$, we need a new reduced binary operator for our $red\_Type$.\\
$red\_bop : binary\_op$ $red\_Type :=
    λ p1$ $p2,  existT Pr (u_S(projT1 (p1) \oplus_S projT1 (p2))) (Pr\_br$ $p1$ $p2).$\\
Here existT is also an auxiliary function that take the element of type $S$ and the proof of the element to form the an element in the reduced set. (which is inverse to projT$_1$ and projT$_2$)\\
Here we assume that our binary operator $\oplus_S$ has such properties separately on $S$:
\begin{itemize}
\item congruence 
\item associativity 
\item commutativity 
\item selective 
\item idempontent 
\item has id $\bar0$ 
\item has annihilator $\bar1$ 
\end{itemize}
It is easy to proof that $red\_bop$ has such properties on $red\_Type$:
\begin{itemize}
\item Lemma red\_bop\_cong : $congruence \rightarrow bop\_congruence$ $red\_Type$ $red\_eq$ $red\_bop.$ 
\item Lemma red\_bop\_ass : $associativity \rightarrow bop\_associative$ $red\_Type$ $red\_eq$ $red\_bop.$ 
\item Lemma red\_bop\_comm : $commutativity \rightarrow bop\_commutative$ $red\_Type$ $red\_eq$ $red\_bop.$
\item Lemma red\_bop\_sel : $selective \rightarrow bop\_selective$ $red\_Type$ $red\_eq$ $red\_bop.$ 
\item Lemma red\_bop\_idem : $idempontent \rightarrow bop\_idempotent$ $red\_Type$ $red\_eq$ $red\_bop.$  
\item Lemma red\_bop\_id : \\$uop\_preserves\_id$ $S$ $=_S$ $\oplus_S$ $u_S \rightarrow\\ bop\_exists\_id$ $S$ $=_S$ $\oplus_S \rightarrow\\ bop\_exists\_id$ $red\_Type$ $red\_eq$ $red\_bop.$ 
\item Lemma red\_bop\_ann : \\$uop\_preserves\_ann$ $S$ $=_S$ $\oplus_S$ $u_S \rightarrow\\ bop\_exists\_ann$ $S$ $=_S$ $\oplus_S \rightarrow\\ bop\_exists\_ann$ $red\_Type$ $red\_eq$ $red\_bop.$ 
\end{itemize}
It is worth mentioning that although the proof in this direction looks very obvious, those proof are not all sufficient. For example, sometimes $red_bop$ can be commutative on $red\_Type$ even if $\oplus_S$ is not commutative on $S$.\\
Fortunately, in our project we only care about whether our reduced set has the same properties when the original problem set has certain properties. So after our construction, we will specify the properties of the original problem set.\\\\
However, as we mentioned in the introduction, there are quite a few problems with this traditional reduction.\\
First of all, the proof part can not be well expressed in the outside world. We must know that many programming languages do not have the ability to represent and prove proposition.\\
Second, when we apply a lot of reduction at the same time, this traditional representation will make our reduced set winding and confusing. For example, suppose we have two reductions $r_1$ and $r_2$, then the reduced problem set that apply reduction 2 after reduction 1 on the problem set must be represent as $\{x \in \{ y \in S \& r_1 (y) = y\} \& r_2 (x) = x\}$. At the same time we need to pay attention that the type of $r_2$ must be $\{ y \in S \& r_1 (y) = y\} \rightarrow \{ y \in S \& r_1 (y) = y\}$, but it should be $S \rightarrow S$ by our definition.\\\\
Because of the above problems, we have to consider other ways to represent our reduction (this is also the purpose and significance of our project).

\section{Semigroup With Direct Reduction}
The previous section mentioned that the purpose of our project is to clarify the relationship between reduction and semiring properties. 
In addition to the semiring's distribution relationship between the two operators, we can first think of it as two semigroups (then consider identity and annihilator issues).
Hence, we need to give a primitive definition on semigroup (Some more detailed definitions will be added later as needed).\\
Initially, according to the properties of the binary relationship, we define a record that contains those properties:
\begin{lstlisting}[]
Record eqv_proofs (S : Type) (eq : brel S) :=
{
  eqv_reflexive      : brel_reflexive S eq            
; eqv_transitive     : brel_transitive S eq           
; eqv_symmetric      : brel_symmetric S eq
; eqv_congruence     : brel_congruence S eq eq                                      
; eqv_witness        : S                                      
}.
\end{lstlisting}
The definition includes the four properties: reflexive, transitive, symmetric and congruence for a binary relationship on a given type that I mentioned above. 
At the same time, in order to ensure that our definition of semigroup is not a trivial semigroup (which means there is at least one element in the semigroup), we have added a witness property.\\
Next we need to define the properties of the binary operator in the semigroup:
\begin{lstlisting}[]
Record semigroup_proofs (S: Type) (eq : brel S) (b : binary_op S) := 
{
  sg_associative   : bop_associative S eq b
; sg_congruence    : bop_congruence S eq b
; sg_commutative_d : bop_commutative_decidable S eq b                                      
}.
\end{lstlisting}
This includes associativity, congruence and commutativity. 
It is worth noting that since commutativity is not a property that semigroups must possess, we hereby conclude that a semigroup can be commutative or not commutative, but this must be decidable.\\
For a given type $S$, binary relationship $=_S$ and binary operator $\oplus_S$,\\
not$\_$commutativity: $\exists a,b\in S, a \oplus_S b \not=_S b \oplus_S a$,\\
commutativity$\_$decidable: $commutative$ $S$ $=_S$ $\oplus_S$ $+$ $not\_commutative$ $S$ $=_S$ $\oplus_S$.\\

Finally we can define our semigroup record as:
\begin{lstlisting}[]
Record semigroup (S : Type) :=
{
   eq   : brel S      
;  bop  : binary_op S
;  eqv  : eqv_proofs S eq               
;  sgp  : semigroup_proofs S eq bop
}.

\end{lstlisting}
According to our previous definition of traditional reduction and its properties, and the semigroup record we defined, we can try to define our reduction semigroup.
But first, because our reduction is a traditional reduction, we need to define the properties of reduction together as a record.
\begin{lstlisting}[]
Record reduction_eqv_proofs (S : Type) (eq : brel S) 
(r : unary_op S) :=
{
  rep_cong  : uop_congruence S eq r 
; rep_idem  : uop_idempotent S eq r 
}.

Record reduction_bop_proofs (S : Type) (eq : brel S) 
(r : unary_op S) (b : binary_op S) :=
{
  rb_left  : bop_left_uop_invariant S eq b r
; rb_right : bop_right_uop_invariant S eq b r
}.
\end{lstlisting}
The definitions of the two blocks contain two properties $\forall A \in S, reduce(reduce(A)) = reduce(A)$ and $\oplus : S \times S \rightarrow S, \forall A,B \in S, reduce(reduce(A)\oplus B) = reduce(A\oplus B) = reduce(A\oplus reduce(B))$ respectively, and it is also the definition of the reduction mentioned earlier by Ahnont Wongseelashote in his article \cite{WONGSEELASHOTE197955}.\\\\

For a given type $S$, binary relationship $eqS$, unary operator $r$, binary operator $b$, if we have a semigroup $sg$ : semigroup $S$ on $eqS$ and $b$, then we can extract the $eqv\_proof$ from $sg$ as $eqv$, and we should initially construct the $eqv\_proof$ on the $red\_Type$ (reduction type). 
\begin{lstlisting}
Definition reduced_eqv_proofs : 
(S : Type) (eq : brel S) (r : unary_op S) (b : binary_op S)
	(eqv : eqv_proofs S eq),  
	reduction_eqv_proofs S eq r -> 
	eqv_proofs (red_Type S r eq) (red_eq S r eq)
\end{lstlisting}
It is worth mentioning that if the unary operator $r$ is a reduction, then we assume that $r$ has the properties of reduction.\\

Next, we extract the semigroup proof $csg$ from $sg$ and we construct the semigroup proof for $red\_Type$:
\begin{lstlisting}
Definition reduced_semigroup_proofs :
    (S : Type)
    (eq : brel S)
    (r : unary_op S)
    (b : binary_op S)
    (eqv : eqv_proofs S eq)
    (csg : semigroup_proofs S eq b)
    (req : reduction_eqv_proofs S eq r)     
    (rb  : reduction_bop_proofs S eq r (bop_reduce r b)) 
    (dec : bop_commutative_decidable (red_Type S r eq) 
    	(red_eq S r eq) (red_bop S b r eq (rep_idem _ _ _ req))), 
     semigroup_proofs (red_Type S r eq) 
     	(red_eq S r eq) (red_bop S b r eq (rep_idem _ _ _ req))
\end{lstlisting}
It is worth noting here that since we need to give the new semigroup we defined about the properties of commutative, we have to know whether operator $red\_bop$ after reduction is commutative on $red\_Type$, but for the most time this is not an easy task.\\\\

Finally, we can define a semigroup on $red\_Type$ following the traditional reduction.
\begin{lstlisting}
Definition semigroup_reduced:
     (S : Type)
     (csg : semigroup S)
     (r : unary_op S)
     (req : reduction_eqv_proofs S (eq S csg) r)
     (rb  : reduction_bop_proofs S (eq S csg) r 
     	(bop_reduce r (bop S csg)))
     (dec : bop_commutative_decidable (red_Type S r (eq S csg)) 
     	(red_eq S r (eq S csg)) (red_bop S (bop S csg) r 
     	(eq S csg) (rep_idem _ _ _ req))),      
           semigroup (red_Type S r (eq S csg))
\end{lstlisting}

There are two obvious problems with defining the reduction semigroup in this way.\\
1, As mentioned above, even if we know that the original semigroup is commutative or not, it is difficult to infer whether $red\_bop$ is commutative under $red\_Type$.
According to our previous proof we know that if $b$ is commutative on $S$, then $red\_bop$ is also commutative on $red\_Type$.
However, even if b is not commutative on S, $red\_bop$ may still be commutative on $red\_Type$. This proof will become very complicated, and it is very dependent on the functionality of reduction and $b$. It is difficult to be generalized.\\
As mentioned in the previous section, for the most time we actually relate to our semigroup "is" commutative and not concerned with it "is whether or not" commutative.
So we can add the properties of the commutative to our defined record:
\begin{lstlisting}
Record commutative_semigroup_proofs (S: Type) 
(eq : brel S) (b : binary_op S) := 
{
  csg_associative   : bop_associative S eq b
; csg_congruence    : bop_congruence S eq b
; csg_commutative   : bop_commutative S eq b                                      
}.

Record commutative_semigroup (S : Type) :=
{
   ceq   : brel S      
;  cbop  : binary_op S
;  ceqv  : eqv_proofs S ceq               
;  csgp  : commutative_semigroup_proofs S ceq cbop
}.
\end{lstlisting}
Unlike before, here we assume that our semigroup $sg$ is commutative, so it is easy to construct a commutative semigroup on $red\_Type$.

\begin{lstlisting}
Definition reduced_commutative_semigroup_proofs :
  ∀ (S : Type)
    (eq : brel S)
    (r : unary_op S)
    (b : binary_op S)
    (eqv : eqv_proofs S eq)
    (csg : commutative_semigroup_proofs S eq b)
    (req : reduction_eqv_proofs S eq r)     
    (rb  : reduction_bop_proofs S eq r (bop_reduce r b)), 
         commutative_semigroup_proofs (red_Type S r eq) 
         (red_eq S r eq) (red_bop S b r eq (rep_idem _ _ _ req))

Definition commutative_semigroup_direct_reduction :
  ∀ (S : Type)
     (csg : commutative_semigroup S)
     (r : unary_op S)
     (req : reduction_eqv_proofs S (ceq S csg) r)     
     (rb  : reduction_bop_proofs S (ceq S csg) r 
     	(bop_reduce r (cbop S csg))),
         commutative_semigroup (red_Type S r (ceq S csg))
\end{lstlisting}
2. After our construction, although we successfully defined semigroup on $red\_Type$, we lose information about reduction $r$ (the function of $r$ is not defined in the semigroup's record).\\\\
Therefore, the use of traditional reduction and direct definition of reduced semigroups is limited by the limitations of reduction itself and the limitations of semigroup definitions, which is not a good choice, and we need to find a better representation.


\section{Binary Reduced Relationship and Binary Full Reduced Operator}
Let have a review on the definition of the traditional reduction with its binary relationship and binary operator.\\
$Pr (x : S) :=   u_S (x) =_S x$.\\
$red\_Type := \{x : S \& Pr(x)\}$.\\
$red\_eq : red\_Type \rightarrow red\_Type \rightarrow bool 
:= \lambda p_1 p_2, projT_1 (p_1) =_S projT_1 (p_2)$.\\
$red\_bop : binary\_op$ $red\_Type :=
    λ p1$ $p2,  existT Pr (u_S(projT1 (p1) \oplus_S projT1 (p2))) (Pr\_br$ $p1$ $p2)$.\\
We realised that actually when we are using the reduction, we wind it on our binary relationship and binary operator. 
So we try to discard the proof part and define new $eq\_r$ and $b\_r$ according to $eq$, $r$ and $b$.\\
Definition $brel\_reduce : brel$ $S := \lambda x$ $y,  eq$ $(r$ $x)$ $(r$ $y).$ \\
Definition $bop\_reduce : binary\_op$ $S := \lambda x$ $y,  r$ $(b$ $x$ $y).$\\
$y).$\\   
Definition $bop\_full\_reduce : binary\_op$ $S := \lambda x$ $y,  r(b$ $(r$ $x)$ $(r$ $y)).$\\
We enwrap reduction on equality to define a new binary relationship. 
We use the same method and define two new binary operators based on the difference in the reduction placement.
However, in fact, according to the properties of the left$\_$invariant and right$\_$invariant of reduction mentioned earlier, the two binary operators are the same.\\\\
Next we need to prove that by using of this representation of reduction, whether the properties of the binary operator $bS$ we mentioned earlier is still valid on $S$.
Here we assume that our binary operator $eqS$ has such properties separately on $S$:
\begin{itemize}
\item congruence 
\item associativity 
\item commutativity 
\item selective 
\item idempontent 
\item has id $\bar0$ 
\item has annihilator $\bar1$ 
\end{itemize}
And the reduction $r$ (unary operator) has the properties of congruence, idempontent, left/right invariant.\\
It is easy to proof that $bop\_full\_reduce$ $r$ $bS$ has such properties on $brel\_reduce$ $r$ $eqS$:
\begin{itemize}
\item Lemma bop\_full\_reduce\_congruence : \\$congruence \rightarrow congruence$ $S$ $(brel\_reduce$ $r$ $eqS)$ $(bop\_full\_reduce$ $r$ $bS).$ 
\item Lemma bop\_full\_reduce\_associative : \\$associative \rightarrow associative$ $S$ $(brel\_reduce$ $r$ $eqS)$ $(bop\_full\_reduce$ $r$ $bS).$ 
\item Lemma bop\_full\_reduce\_commutative : \\$commutative \rightarrow commutative$ $S$ $(brel\_reduce$ $r$ $eqS)$ $(bop\_full\_reduce$ $r$ $bS).$ 
\item Lemma bop\_full\_reduce\_idempotent : \\$idempotent \rightarrow idempotent$ $S$ $(brel\_reduce$ $r$ $eqS)$ $(bop\_full\_reduce$ $r$ $bS).$ 
\item Lemma bop\_full\_reduce\_selective : \\$selective \rightarrow selective$ $S$ $(brel\_reduce$ $r$ $eqS)$ $(bop\_full\_reduce$ $r$ $bS).$  
\item Lemma bop\_full\_reduce\_exists\_id : \\$uop\_preserves\_id$ $S$ $eqS$ $bS$ $u_S \rightarrow\\ bop\_exists\_id$ $S$ $eqS$ $bS \rightarrow\\ bop\_exists\_id$ $S$ $(brel\_reduce$ $r$ $eqS)$ $(bop\_full\_reduce$ $r$ $bS).$ 
\item Lemma bop\_full\_reduce\_exists\_ann : \\$uop\_preserves\_ann$ $S$ $eqS$ $bS$ $u_S \rightarrow\\ bop\_exists\_ann$ $S$ $eqS$ $bS \rightarrow\\ bop\_exists\_ann$ $S$ $(brel\_reduce$ $r$ $eqS)$ $(bop\_full\_reduce$ $r$ $bS).$ 
\end{itemize}
We find that we only use the left/right invariant properties of reduction to prove the property of the associative.
Thus, maybe there is more than one way to axiomatise a reduction. 
In the following chapters we will introduce a new associative: the pseudo associative, and use it to prove the properties of reductions.
\subsection{Isomorphic to Traditional Reduction}
However, in fact it is not enough to justify the correctness of this new representation by establishment of these properties. We need to establish some isomorphisms to prove that this new reduction (brel\_reduce and bop\_full\_reduce) is equivalent to the $red_Type$ $red_eq$ and $red_bop$.
\begin{itemize}
\item Lemma red\_ref\_iso : brel\_reflexive red\_Type red\_eq $\leftrightarrow$ brel\_reflexive S (brel\_reduce r eqS).
\item Lemma red\_sym\_iso : brel\_symmetric red\_Type red\_eq $\leftrightarrow$ brel\_symmetric S (brel\_reduce r eqS).
\item Lemma red\_tran\_iso : brel\_transitive red\_Type red\_eq $\leftrightarrow$ brel\_transitive S (brel\_reduce r eqS).
\item Lemma red\_cong\_iso : bop\_congruence red\_Type red\_eq red\_bop $\leftrightarrow$ bop\_congruence S (brel\_reduce r eqS) (bop\_full\_reduce r b).
\item Lemma red\_bop\_ass\_iso : bop\_associative red\_Type red\_eq red\_bop $\leftrightarrow$ bop\_associative S eqS r b.
\item Lemma red\_comm\_iso :  bop\_commutative red\_Type red\_eq red\_bop $\leftrightarrow$ bop\_commutative S (brel\_reduce r eqS) (bop\_full\_reduce r b).
\item Lemma red\_exists\_id\_iso :  bop\_exists\_id red\_Type red\_eq red\_bop $\leftrightarrow$ bop\_exists\_id S (brel\_reduce r eqS) (bop\_full\_reduce r b).
\end{itemize}
Based on these proofs, we can conclude that binary reduced relationship and binary full reduced operator have the same properties as traditional reduction. They are both a form of reduction.\\
This form of reduction can be well expressed in the outside world in terms of the unary operator.
At the same time, we can also avoid (ease) the winding problem between the reduction and original problem set.
\section{Reduced Semigroup}
In the section of direct reduction, we defined the semigroup, and discovered that the direct reduced semigroup would lose the reduction information.
Now we try to write reduction as a definition in the record of the semigroup.
\begin{lstlisting}
Record rsemigroup (S : Type) :=
{
   r_eq   : brel S
;  r_rep  : unary_op S                      
;  r_bop  : binary_op S
;  r_eqv  : eqv_proofs S r_eq
;  r_rpv  : reduction_eqv_proofs S r_eq r_rep                        
;  r_sgp  : semigroup_proofs S r_eq r_bop
}.

Record commutative_rsemigroup (S : Type) :=
{
   cr_eq   : brel S
;  cr_rep  : unary_op S                      
;  cr_bop  : binary_op S
;  cr_eqv  : eqv_proofs S cr_eq
;  cr_rpv  : reduction_eqv_proofs S cr_eq cr_rep                        
;  cr_sgp  : commutative_semigroup_proofs S cr_eq cr_bop
}.
\end{lstlisting}
These two are rsemigroups that do not contain commutative properties and rsemigroups that contain commutative properties.
Based on our previous experience, most of the time we will choose to directly use rsemigroups that contain commutative properties.\\
We also define the conversion from rsemigroup to semigroup (actually omission the reduction part and its associated proof).
\begin{lstlisting}
Definition rsemigroup_to_semigroup (S : Type): 
rsemigroup S -> semigroup S
:= λ rsg,
{|    
   eq   := r_eq S rsg 
;  bop  := r_bop S rsg 
;  eqv  := r_eqv S rsg 
;  sgp  := r_sgp S rsg
|}.

Definition commutative_rsemigroup_to_commutative_semigroup 
(S : Type): 
commutative_rsemigroup S -> commutative_semigroup S
:= λ crsg,
{|    
   ceq   := cr_eq S crsg 
;  cbop  := cr_bop S crsg 
;  ceqv  := cr_eqv S crsg 
;  csgp  := cr_sgp S crsg
|}.
\end{lstlisting}

\subsection{Product Semigroup/RSemigroup}
Similar as our direct reduced semigroup that defined in the previous section, it is really easy to construct an rsemigroup for a given type, binary relationship, unary operator (reduction) and binary operator. For some extent, defining an rsemigroup is even easier than we previously defined the direct reduced semigroup, because we show the reduction information obviously in the definition record. Further more we have conversion from rsemigroup to semigroup. In this process, the function omits the information of reduction and its related proof. However, those information and proofs are still be used in the definition of other properties (like commutative). Hence we can say that rsemigroup and semigroup are equivalent in most parts.\\\\
The next step we need to focus on is how to combine two semigroups together.
As I mentioned in the introduction, the CAS system will combine different semigroups (semirings) together (using direct product of lexicographic product) and discuss their characteristics. The example provided in the background part shows that the widest-shortest problem contains two different semirings and uses lexigraphic product to combine them together. Hence, we need to discuss on the combination of two semigroup at first.\\\\
Initially we need to define the product of two eqv\_proof (the proof of properties of the binary relationship).\\
Definition eqv\_proofs\_product : $  eqv\_proofs$ $S$ $eqS \rightarrow eqv\_proofs$ $T$ $eqT \rightarrow eqv\_proofs$ $(S * T)$ $(brel_product$ $eqS$ $eqT)$.\\\\
Then we need to define the product of two semigroup\_proof (the proof of properties of the binary operator).\\
Definition sg\_proofs\_product : $(semigroup\_proofs$ $S$ $eqS$ $bS) \rightarrow (semigroup\_proofs$ $T$ $eqT$ $bT)
               \rightarrow semigroup\_proofs$ $(S * T)$ $(brel\_product$ $eqS$ $eqT )$ $(bop\_product$ $bS$ $bT)$.\\
Because we also have a commutative version of semigroup proof, then we can derive a similar construction.\\
Definition sg\_proofs\_commutative\_product : $(commutative\_semigroup\_proofs$ $S$ $eqS$ $bS) \rightarrow (commutative\_semigroup\_proofs$ $T$ $eqT$ $bT)
               \rightarrow commutative\_semigroup\_proofs$ $(S * T)$ $(brel\_product$ $eqS$ $eqT )$ $(bop\_product$ $bS$ $bT)$.\\\\
Next we can define a simple semigroup combinator for our direct semigroup.\\
Definition semigroup\_product : $semigroup$ $S \rightarrow semigroup$ $T \rightarrow semigroup$ $(S * T)$.\\
Definition commutative\_semigroup\_product : $commutative\_semigroup$ $S \rightarrow commutative\_semigroup$ $T \rightarrow commutative\_semigroup$ $(S * T)$.\\

Furthermore, because our rsemigroup has the information of reduction, it need the proof of product of two reduction proofs.\\
Definition reduction\_eqv\_proofs\_product :$ 
  reduction\_eqv\_proofs$ $S$ $eqS$ $repS \rightarrow
  reduction\_eqv\_proofs$ $T$ $eqT$ $repT \rightarrow
      reduction\_eqv\_proofs$ $(S * T)$ $(brel\_product$ $eqS$ $eqT)$ $(uop\_product$ $ repS$ $repT)$\\\\
Finally we can define our new combinator, the product of two rsemigroup.\\
Definition rsemigroup\_product : $rsemigroup$ $S \rightarrow rsemigroup$ $T \rightarrow rsemigroup$ $S * T)$.\\\\
The appearance of this combinator means that we can combine two different semiring constructed on different reduction together to form a new semigroup. Next we have to focus on how to construct our reduction.
\section{Annihilator Reduction}
Next we have to focus on how to construct our reduction and we need to give some practical examples of reduction.
Following the idea of the widest-shortest problem, we want to eliminate the element (edge) that has 0 capacity in our calculation. 
Recalling our widest-shortest problem, each element is represented in such a form as (capacity, length), or a similar form. Therefore we hope to eliminate such a pair of element as (0, x). Considering that 0 plays a role of annihilator of the additive operator in the widest path problem, which is the first part of the product (actually lexicographic product) semiring, we construct a reduction in the same way.\\
Definition bop\_reduce\_annihilators $\{S$ $T : Type\}$ $(aS : S)$ $(aT : T)$ $(eqS : brel$ $S)$ $(eqT : brel$ $ T) : unary\_op$ $ (S * T)
  := \lambda p, let (s, t) := p$ $ in $ $if$ $ orb $ $(eqS$ $ s $ $aS) $ $(eqT $ $t $ $aT)$ $ then $ $(aS, aT) $ $else p.$
In bop\_reduce\_annihilators, suppose $aS$ is the annihilator of $S$ and $aT$ is the annihilator of $T$, then this reduction will treat all pairs $(s,aT)$ and $(aS,t)$ as $(aS,aT)$.\\
As we mentioned in the definition of reduction, we need to prove the reduction properties of this new reduction.
\begin{lstlisting}
Lemma bop_reduce_annihilators_idempotent 
(S T : Type) (aS : S) (aT : T) (eqS : brel S) (eqT : brel T) :
  brel_reflexive S eqS -> brel_reflexive T eqT -> 
  uop_idempotent (S * T) (brel_product eqS eqT) 
  	(bop_reduce_annihilators aS aT eqS eqT).

Lemma bop_reduce_annihilators_congruence 
(S T : Type) (aS : S) (aT : T) (eqS : brel S) (eqT : brel T) :
  brel_reflexive S eqS -> brel_symmetric S eqS -> 
  brel_transitive S eqS -> 
  brel_reflexive T eqT -> brel_symmetric T eqT -> 
  brel_transitive T eqT -> 
  uop_congruence (S * T) (brel_product eqS eqT) 
  	(bop_reduce_annihilators aS aT eqS eqT).
  	
Lemma bop_reduce_annihilators_left_invariant
      (S T : Type)(aS : S)(aT : T)
      (eqS : brel S)eqT : brel T)
      (bS : binary_op S)(bT : binary_op T)      
  : brel_reflexive S eqS -> brel_reflexive T eqT -> 
    bop_is_ann S eqS bS aS -> bop_is_ann T eqT bT aT -> 
    bop_left_uop_invariant (S * T)
      (brel_product eqS eqT)
      (bop_full_reduce (bop_reduce_annihilators aS aT eqS eqT) 
      			(bop_product bS bT))
      (bop_reduce_annihilators aS aT eqS eqT).

Lemma bop_reduce_annihilators_right_invariant
      (S T : Type)(aS : S)(aT : T)
      (eqS : brel S)eqT : brel T)
      (bS : binary_op S)(bT : binary_op T)      
  : brel_reflexive S eqS -> brel_reflexive T eqT -> 
    bop_is_ann S eqS bS aS -> bop_is_ann T eqT bT aT ->     
    bop_right_uop_invariant (S * T)
      (brel_product eqS eqT)
      (bop_full_reduce (bop_reduce_annihilators aS aT eqS eqT) 
      			(bop_product bS bT))
      (bop_reduce_annihilators aS aT eqS eqT).
\end{lstlisting}

\section{Pseudo Associative}
In the previous case of annihilator reduction we found that since we had previously proved on the bop\_full\_reduce, so long as we can construct a reduction and prove its congruence, idempontent, left/right invariant properties, then we can easily prove all the properties of the reduction on the binary operator and construct a semigroup based on this reduction.\\
Further we find that in the proof of the properties of the binary operator, we only use the properties of the left/right invariant to prove associative. This means that there is probably another kind of weaker prerequisite that could axiomatise a reduction. In fact, we did find such a precondition.\\
For a given type $S$, binary relationship $=_S$, unary operator (reduction) $u_S$, binary operator $\oplus_S$:\\
Definition bop\_pseudo\_associative $:= \forall s, t, u : S, u_S (b (u_S ((u_S (s)) \oplus_S (u_S (t)))) \oplus_S (u_S (u))) =_S u_S ( (u_S (s)) \oplus_S (u_S ( (u_S (t)) \oplus_S (u_S (u)))))$.\\
At the same time, we can prove that pseudo associative is a sufficient condition for the binary operator to be associative on reduction.
\begin{lstlisting}
Lemma bop_full_reduce_pseudo_associative_implies_associative 
(S : Type) (eqS : brel S) (r : unary_op S) (bS : binary_op S) :
  brel_reflexive S eqS ->
  brel_symmetric S eqS ->  
  brel_transitive S eqS ->   
  uop_idempotent S eqS r ->
  uop_congruence S eqS r ->
  bop_congruence S eqS bS ->
  bop_pseudo_associative S eqS r bS -> 
  bop_associative S (brel_reduce r eqS) (bop_full_reduce r bS). 
         
Lemma bop_full_reduce_associative_implies_pseudo_associative 
(S : Type) (eqS : brel S) (r : unary_op S) (bS : binary_op S) :
  brel_reflexive S eqS ->
  brel_symmetric S eqS ->  
  brel_transitive S eqS ->   
  uop_idempotent S eqS r ->
  uop_congruence S eqS r ->
  bop_congruence S eqS bS ->
  bop_associative S (brel_reduce r eqS) (bop_full_reduce r bS) ->
      bop_pseudo_associative S eqS r bS. 
\end{lstlisting}
In fact, the two are almost equivalent at present.
Therefore, when we later prove that a unary operator meets our reduction criteria, we only need to prove that its congruence, iteratorton, and pseudo associative which can also prove all the remaining properties.
Take the previous annihilator reduction as an example, we can prove that this unary operator is pseudo associative at first.
\begin{lstlisting}
Lemma bop_reduce_annihilators_psedo_associative 
(S T : Type)(aS : S)(aT : T)
(eqS : brel S)(eqT : brel T)
(bS : binary_op S)(bT : binary_op T): 
brel_reflexive S eqS -> brel_reflexive T eqT ->
brel_symmetric S eqS -> brel_symmetric T eqT ->
brel_transitive S eqS -> brel_transitive T eqT -> 
bop_associative S eqS bS ->
bop_associative T eqT bT -> 
bop_congruence S eqS bS ->
bop_congruence T eqT bT -> 
bop_is_ann S eqS bS aS ->
bop_is_ann T eqT bT aT ->
bop_pseudo_associative (S * T) 
	(brel_product eqS eqT) 
	(bop_reduce_annihilators aS aT eqS eqT) (bop_product bS bT) .
\end{lstlisting}
The associative property of annihilator reduction can then also be demonstrated, and the same semigroup construction as above can be performed.\\\\
Here, we have reason to believe that we can make some non-traditional reduction into our study after satisfying the pseudo associative even if they do not meet the left/right invariant property.

\section{Predicate Reduction}
We introduced the annihilator reduction based on the pair of annihilators in the previous section. Here we hope to further define a more flexible reduction. 
We predefine a predicate for our problem set $Variable$ $P : S \rightarrow bool$.
Then we can define our reduction as. if the element satisfies the predicate, then we reduce it to a particular element, otherwise keep the element unchanged.\\
$Definition$ $uop\_predicate\_reduce : \forall \{S : Type\}, 
S \rightarrow (S \rightarrow bool) \rightarrow unary\_op$ $S\\ 
  := \lambda  \{S\}$ $s1$ $P$ $s2,$ $if$ $P$ $s2$ $then$ $s1$ $else$ $s2.$\\
Therefore, we can regard the previous annihilator reduction as a kind of special predicate reduction.\\
$Definition$ $P : (S *T) \rightarrow bool := \lambda p, match$ $p$ $with (s, t) => orb (eqS$ $s$ $aS) (eqT$ $t$ $aT) end.$\\
$Definition$ $uop\_rap : unary\_op (S * T) := uop\_predicate\_reduce (aS, aT)$ $P.$\\
Therefore, we will not discuss the properties of annihilator separately any more but will mainly discuss the properties of predicate reduction in this section.\\\\
Here we add a basic definition $Definition$ $pred (S : Type):= S \rightarrow bool.$\\
We add two new properties for the predicate:\\
$Definition$ $pred\_true (S : Type) (P : pred$ $S) (s : S) 
  := P$ $s = true. $\\
$Definition$ $pred\_congruence (S : Type) (eq : brel$ $S) (P : pred$ $S) 
  := \forall (a$ $b : S), eq$ $a$ $b = true \rightarrow P$ $a = P$ $b.$\\
The above two properties are of great significance in our later proof.
In fact, we will add more properties for predicates according to requirements in the later section, which will be introduced in detail later.\\\\

As mentioned earlier, as we want to prove (or build a) reduction, we need to prove its congruence and idempotent properties.
\begin{lstlisting}
Lemma uop_predicate_reduce_congruence (s : S) :
  pred_congruence S eq P -> 
  uop_congruence S eq (uop_predicate_reduce s P). 
  
Lemma uop_predicate_reduce_idempotent : 
forall (s : S), uop_idempotent S eq (uop_predicate_reduce s P). 
\end{lstlisting}
At the same time if we want to prove that it is a traditional reduction on some binary operator, we need to prove its left/right invariant properties.
However, according to our different definition of predicate and different binary operators, we can construct different traditional/non-traditional reduction using our uop\_predicate\_reduce. 
For example, our later case "min plus with ceiling" has both traditional reduction on 'min' and 'plus' operator. And in the case of "elementary path problem", we have traditional reduction on our 'concatenation' operator, however, the reduction on 'min' operator is not traditional.\\
So we need to follow another way to represent our predicate (given our predicates certain restrictions/properties), and here we define the following three properties based on our predicate.\\
$Definition$ $pred\_bop\_decompose (S : Type) (P : pred$ $S) (bS : binary\_op$ $S) 
  := \forall (a$ $b : S), P$ $(bS$ $a$ $b) = true \rightarrow (P$ $a = true) + (P$ $b = true).$\\
$Definition$ $pred\_bop\_compose (S : Type) (P : pred$ $S) (bS : binary\_op$ $S) 
  := \forall (a$ $b : S), (P$ $a = true) + (P$ $b = true) \rightarrow P (bS$ $a$ $b) = true.$\\
$Definition$ $pred\_preserve\_order (S : Type) (P : pred$ $S) (eqS : brel$ $S) (bS : binary\_op$ $S)
  := \forall (a$ $b : S), eqS (bS$ $a$ $b) a = true \rightarrow P$ $a = true \rightarrow P$ $b = true.$
Next we will use "min plus with ceiling" as an example to explain these three properties of predicate, of course, in the following section we will also systematically define "min plus with ceiling" problem.\\
In the ceiling problem, we will set a ceiling in advance, and then reduce all elements larger than ceiling (here we define the problem set as the natural number) to reduce to ceiling. \\
Hence our min operator has the property of decompose, which means if the minimum of two number satisfies the predicate (larger than the ceiling), then there will be at least one of them satisfies the predicate (actually both of them should satisfied the predicate).\\
And our plus operator has the property of composes, which means if one of the element is larger than the ceiling, then the sum of the two number will larger the the ceiling.\\
Further more, our min operator will satisfies the property of preserve order, which means if we have orders between elements, then $ a \leq b$ (represents as $a \oplus b = a$) $\rightarrow P$ $a \rightarrow P$ $b$, which means if the minimum value of the two elements satisfies the predicated, then another one must satisfy the predicate.\\\\
After defining those properties, we can talk about the properties of our predicate reduction.\\
We find that, if the operator has properties of either "compose" or "preserve order", then the reduction on it will be a traditional reduction. And this also directly explains why predicates reduce in "min plus with ceiling" problem is traditional on both two operator and it is also traditional on the concatenation in "elementary path" problem, but not tradition on the "min/select" operator in "elementary path" problem.\\

\begin{lstlisting}
Lemma bop_left_uop_invariant_predicate_reduce :
  forall (s : S) (bS : binary_op S),
    pred_true S P s -> 
    pred_bop_compose S P bS ->
    bop_left_uop_invariant S eq 
    		(bop_reduce (uop_predicate_reduce s P) bS) 
    		(uop_predicate_reduce s P).
         
Lemma bop_left_uop_invariant_predicate_reduce_v2 :
  forall (s : S) (bS : binary_op S),
    pred_true S P s ->
    pred_congruence S eq P -> 
    bop_selective S eq bS ->
    bop_is_id S eq bS s ->        
    pred_preserve_order S P eq bS ->
    bop_left_uop_invariant S eq 
    		(bop_reduce (uop_predicate_reduce s P) bS) 
    		(uop_predicate_reduce s P).

Lemma bop_right_uop_invariant_predicate_reduce :
  ∀ (s : S) (bS : binary_op S),
    pred_true S P s -> 
    pred_bop_compose S P bS ->    
    bop_right_uop_invariant S eq 
    		(bop_reduce (uop_predicate_reduce s P) bS) 
    		(uop_predicate_reduce s P).
        
Lemma bop_right_uop_invariant_predicate_reduce_v2 :
  forall (s : S) (bS : binary_op S),
    pred_true S P s ->
    pred_congruence S eq P -> 
    bop_selective S eq bS ->
    bop_commutative S eq bS ->
    bop_is_id S eq bS s ->        
    pred_preserve_order S P eq bS ->
    bop_right_uop_invariant S eq 
    		(bop_reduce (uop_predicate_reduce s P) bS) 
    		(uop_predicate_reduce s P).
\end{lstlisting}
The next thing we need to care about is the properties of associative.
As we mentioned earlier, there are two ways to prove that the operator under reduction is still associative. One is to prove that this reduction is traditional in this operator. The second is to prove that this operator and reduction have the pseudo-associative properties.
So we can first prove that our operator is compose, or preserve order, then our operator is associative under predicate reduction.
\begin{lstlisting}
Lemma bop_associative_fpr_compositional :
  forall (s : S) (bS : binary_op S),
    pred_true S P s    -> 
    pred_congruence S eq P ->     
    pred_bop_compose S P bS ->
    bop_congruence S eq bS ->         
    bop_associative S eq bS ->
    bop_associative S 
        (brel_reduce (uop_predicate_reduce s P) eq) 
        (bop_fpr s P bS).
\end{lstlisting}
At the same time, we find that if our operator is decompose and the element (s) in reduction is the identity/annihilator of operator, we can prove the nature of pseudo associative.
\begin{lstlisting}
Lemma bop_pseudo_associative_fpr_decompositional_id :
  forall (c : S) (bS : binary_op S),
    pred_true S P c ->
    pred_congruence S eq P ->
    bop_congruence S eq bS ->     
    bop_associative S eq bS -> 
    pred_bop_decompose S P bS ->
    bop_is_id S eq bS c -> 
    bop_pseudo_associative S eq (uop_predicate_reduce c P) bS. 
    
Lemma bop_associative_fpr_decompositional_id : 
  forall (c : S) (bS : binary_op S),
    pred_true S P c -> 
    pred_congruence S eq P -> 
    bop_congruence S eq bS ->     
    bop_associative S eq bS ->
    pred_bop_decompose S P bS ->
    bop_is_id S eq bS c -> 
    bop_associative S 
    		brel_reduce (uop_predicate_reduce c P) eq) 
    		(bop_fpr c P bS). 
    		
Lemma bop_pseudo_associative_fpr_decompositional_ann :
  forall (s : S) (bS : binary_op S),
    pred_true S P s -> 
    pred_congruence S eq P ->
    bop_associative S eq bS ->    
    pred_bop_decompose S P bS ->
    bop_is_ann S eq bS s ->     
    bop_pseudo_associative S eq (uop_predicate_reduce s P) bS.
    
Lemma bop_associative_fpr_decompositional_ann : 
  forall (c : S) (bS : binary_op S),
    pred_true S P c -> 
    pred_congruence S eq P -> 
    bop_congruence S eq bS ->     
    bop_associative S eq bS ->
    pred_bop_decompose S P bS ->
    bop_is_ann S eq bS c -> 
    bop_associative S 
    		(brel_reduce (uop_predicate_reduce c P) eq) 
    		(bop_fpr c P bS). 
\end{lstlisting}

In fact, during the process of our proof, we found that for a semiring in general, its additive component is decompose, and its multiplicative component is  compose. 
And because the element (s) in our reduction always be the identity of the additive component, we can conclude that our predicate reduction is associative in most cases.\\

The rest of the properties we can prove them by using lemmas in the bop \_full \_reduce use.

\section{Discussion on Distributivity}
However, as mentioned in the previous chapter, our project is not only discussing the relationship between reduction and semigroup, but it will be a further focus on some of the properties of semiring. 
Therefore, we need to pay attention to the respective properties of the additive and multiplicative components of the semiring, and the properties associated with them.
For example, in the predicate reduce of our previous chapter, the element of the reduction (s) is usually the identity/annihilator of operator. However, in semiring, the identity of additive component is also an annihilator of multiplicative component, and the annihilator of additive component is also an identity of the multiplicative component.
What's more important is the distributivity between the two operators.
Here we first prove the distribution law of our predicate reduce.
\begin{lstlisting}
Lemma bop_fpr_left_distributive :
  forall (s : S) (add mul : binary_op S),
     pred_true S P s -> 
     pred_congruence S eq P ->
     pred_bop_decompose S P add ->
     pred_bop_decompose S P mul ->          
     bop_congruence S eq add ->     
     bop_congruence S eq mul -> 
     bop_is_id S eq add s ->     
     bop_is_ann S eq mul s ->
     bop_left_distributive S eq add mul ->
     bop_left_distributive S 
     	(brel_reduce (uop_predicate_reduce s P) eq) 
     	(bop_fpr s P add) (bop_fpr s P mul).
       
Lemma bop_fpr_right_distributive :
  forall (s : S) (add mul : binary_op S),
     pred_true S P s -> 
     pred_congruence S eq P ->
     pred_bop_decompose S P add ->
     pred_bop_decompose S P mul ->          
    bop_congruence S eq add ->     
    bop_congruence S eq mul -> 
    bop_is_id S eq add s ->     
    bop_is_ann S eq mul s ->
    bop_right_distributive S eq add mul ->
    bop_right_distributive S 
    		(brel_reduce (uop_predicate_reduce s P) eq) 
    		(bop_fpr s P add) (bop_fpr s P mul).
\end{lstlisting}
However, as we mentioned before, our multiplicative components are generally compose rather than decompose, so we need to prove the distributive law in another way.
\begin{lstlisting}
Lemma bop_fpr_left_distributive_v2 :
  forall (s : S) (add mul : binary_op S),
     pred_true S P s -> 
     pred_congruence S eq P ->
     pred_bop_decompose S P add ->
     pred_preserve_order S P eq add ->
     bop_congruence S eq add ->     
     bop_congruence S eq mul -> 
     bop_selective S eq add ->
     bop_commutative S eq add ->
     bop_is_id S eq add s ->     
     bop_is_ann S eq mul s ->
     bop_left_distributive S eq add mul ->
     bop_left_distributive S 
     	(brel_reduce (uop_predicate_reduce s P) eq) 
     	(bop_fpr s P add) (bop_fpr s P mul).
     	
Lemma bop_fpr_right_distributive_v2 :
  forall (s : S) (add mul : binary_op S),
     pred_true S P s -> 
     pred_congruence S eq P ->
     pred_bop_decompose S P add ->
     pred_preserve_order S P eq add ->
     bop_congruence S eq add ->     
     bop_congruence S eq mul -> 
     bop_selective S eq add ->
     bop_commutative S eq add ->
     bop_is_id S eq add s ->     
     bop_is_ann S eq mul s ->
     bop_right_distributive S eq add mul ->
     bop_right_distributive S 
     	(brel_reduce (uop_predicate_reduce s P) eq) 
     	(bop_fpr s P add) (bop_fpr s P mul).
\end{lstlisting}
Therefore, we find that our additive component is decompose and preserve order, then our semiring is distributive.\\
However, this kind of proof is ultimately too specific. We hope to find out whether there is a correlation between reduction itself and distributive property.\\\\
Initially, similar to the property of pseudo-associative, we define two new properties called pseudo-left/right-distributivitt.\\
$Definition$ $bop\_pseudo\_left\_distributive (S : Type) (eq : brel$ $S) (r : unary\_op$ $S) (add$ $mul : binary\_op$ $S) \\
  := \forall a$ $b$ $c : S, 
  eq (r (mul (r$ $a) (r (add (r$ $b) (r$ $c))))) (r (add (r (mul (r$ $a) (r$ $b))) (r (mul (r$ $a) (r$ $c))))) = true. $
         
$Definition$ $bop\_pseudo\_right\_distributive (S : Type) (eq : brel$ $S) (r : unary\_op$ $S) (add$ $mul : binary\_op S) \\
  := \forall a$ $b$ $c : S, 
  eq (r (mul (r (add (r$ $b) (r$ $c))) (r$ $a))) (r (add (r (mul (r$ $b) (r$ $a))) (r (mul (r$ $c) (r$ $a))))) = true.$
Then we were surprised to find that pseudo-distributive is equivalent to that the reduction is traditional on the two operators.
\begin{lstlisting}
Lemma red_bop_left_dist_iso : 
bop_left_distributive T eqT addT mulT <-> 
bop_pseudo_left_distributive S eq r add mul. 

Lemma red_bop_right_dist_iso : 
bop_right_distributive T eqT addT mulT <-> 
bop_pseudo_right_distributive S eq r add mul.
\end{lstlisting}
Going further, we are even more surprised to find that the property of the pseudo-distributive we have just defined is equivalent to the real distributive on reduction.
\begin{lstlisting}
Lemma bop_reduce_pseudo_left_distributivity_iso :
bop_left_distributive S (brel_reduce r eq) 
	(bop_reduce_args r add) (bop_reduce_args r mul) <->
bop_pseudo_left_distributive S eq r add mul.

Lemma bop_reduce_left_distributivity_iso :
bop_left_distributive S (brel_reduce r eq) 
	(bop_reduce_args r add) (bop_reduce_args r mul)
                     <->
bop_left_distributive S (brel_reduce r eq) 
	(bop_full_reduce r add) (bop_full_reduce r mul).

Lemma bop_reduce_pseudo_right_distributivity_iso :
bop_right_distributive S (brel_reduce r eq) 
	(bop_reduce_args r add) (bop_reduce_args r mul) <->
bop_pseudo_right_distributive S eq r add mul.

Lemma bop_reduce_right_distributivity_iso :
bop_right_distributive S (brel_reduce r eq) 
	(bop_reduce_args r add) (bop_reduce_args r mul)
                     <->
bop_right_distributive S (brel_reduce r eq) 
	(bop_full_reduce r add) (bop_full_reduce r mul).
\end{lstlisting}

Therefore, we can conclude that as of now, the properties of the distributivity and the reduction that is traditional on the two operators in the semiring are equivalent.
\section{Simple example: Min Plus with Ceiling Semiring}

\section{Add Constant by Disjoint Union}

\section{Predicate Reduction with Disjoint Annihilator}

\section{Another example: Elementary Path Problem}


\chapter{Evaluation} 

%For any practical projects, you should almost certainly have
%some kind of evaluation, and it's often useful to separate 
%this out into its own chapter. \\\\\\


\chapter{Summary and Conclusions} 

%As you might imagine: summarizes the dissertation, and draws 
%any conclusions. Depending on the length of your work, and 
%how well you write, you may not need a summary here. 
%
%You will generally want to draw some conclusions, and point
%to potential future work. 




\appendix
\singlespacing

\bibliographystyle{unsrt} 
\bibliography{MphilProject}

\end{document}
