%% 
%% ACS project dissertation template. 
%% 
%% Currently designed for printing two-sided, but if you prefer to 
%% print single-sided just remove ",twoside,openright" from the 
%% \documentclass[] line below. 
%%
%%
%%   SMH, May 2010. 


\documentclass[a4paper,12pt,twoside,openright]{report}


%%
%% EDIT THE BELOW TO CUSTOMIZE
%%

\def\authorname{Zongzhe Yuan\xspace}
\def\authorcollege{Christ's College\xspace}
\def\authoremail{zy272@cl.cam.ac.uk}
\def\dissertationtitle{Extending CAS with Algebraic Reductions}
\def\wordcount{14,235}


\usepackage{epsfig,graphicx,parskip,setspace,tabularx,xspace} 
\usepackage{a4wide,parskip,times}
\usepackage{a4wide,parskip}
\usepackage{amsmath}
\usepackage{nicefrac}
\usepackage{ amssymb }
\usepackage{mleftright}

%% START OF DOCUMENT
\begin{document}


%% FRONTMATTER (TITLE PAGE, DECLARATION, ABSTRACT, ETC) 
\pagestyle{empty}
\singlespacing
\input{titlepage}
\onehalfspacing
\newpage
{\Huge \bf Declaration}

\vspace{24pt} 

I \authorname of \authorcollege, being a candidate for the M.Phil in
Advanced Computer Science, hereby declare that this report and the
work described in it are my own work, unaided except as may be
specified below, and that the report does not contain material that
has already been used to any substantial extent for a comparable
purpose.

The project and dissertation are finished under the guidance of my supervisor Dr Griffin.

The appendix contains the additional verbose elements of the Coq proof content, which is not included into the main thesis, and is only used to provide a better understanding for the readers.

\vspace{24pt}
Total word count: \wordcount \\
(excluding the words from appendix)

\vspace{60pt}
\textbf{Signed}: 

\vspace{12pt}
\textbf{Date}:


\vfill

This dissertation is copyright \copyright 2018 \authorname. 
\\
All trademarks used in this dissertation are hereby acknowledged.



\newpage
\vspace*{\fill}

\singlespacing
\newpage
{\Huge \bf Abstract}
\vspace{24pt} 

%This topic is originated from the expansion of a routing problem that was not specifically reasoned in the L11 class. We classify this class of problems as a specific type of problem, reduction. Although we discussed some of the properties of reduction in L11, however, the construction of the reduction for what we need in the final construction of path problem will conflict with the definition of reduction itself. Therefore, it shows that our definition of the reduction does not resolve our actual problem.
%
%When we traced the source back, we found the definition of reduction from a paper by Wongseelashote in 1979. Although Wongseelashote very skilfully proposed the concept of reduction when discussing path problems, it is regrettable that this paper did not provide detailed structural proof of reduction for its properties.
%
%Therefore, our project is comprised of the following three parts. First of all, the classical reduction is expressed, and we are trying to reason the properties of reduction itself. Next, we will try to represent/define reduction in another way, not only to facilitate the implementation, but also to decrease the limitation of the reduction definition to practical problems. That generalised reduction in good representation could help us defining the reduction that we need in the problem. After that, we will define a kind of reduction according to our requirements, and use this kind of reduction to construct numerous reduction for realistic examples. Finally, we will use these practical examples as a combination to resolve the path problem that was mentioned in L11 lecture.

This dissertation presents an in-depth exploration of algebraic reductions first introduced by Wongseelashote in 1979. All of our mathematical work was done using the Coq assistant prover. However, the dissertation presents our results in traditional informal mathematics. Our Coq implementation will be integrated into Dr Griffin's Coq code for Combinators of Algebraic Systems (CAS).

\newpage
\vspace*{\fill}


\pagenumbering{roman}
\setcounter{page}{0}
\pagestyle{plain}
\tableofcontents
\listoffigures
\listoftables

\onehalfspacing

%% START OF MAIN TEXT 

\chapter{Introduction}
\pagenumbering{arabic} 
\setcounter{page}{1} 

This is the introduction where you should introduce your work.  In
general the thing to aim for here is to describe a little bit of the
context for your work --- why did you do it (motivation), what was the
hoped-for outcome (aims) --- as well as trying to give a brief
overview of what you actually did.

It's often useful to bring forward some ``highlights'' into 
this chapter (e.g.\ some particularly compelling results, or 
a particularly interesting finding). 

It's also traditional to give an outline of the rest of the
document, although without care this can appear formulaic 
and tedious. Your call. 

\section{Introduction to the Path Problem and Algebraic Solution}


The initial point of the project is the algebraic path problem. 
At the very beginning, programmer and scientists designed algorithms to solve each of the path problem. The most famous path problem is the shortest distance problem and there are several well-known algorithm that can solve such a problem: Dijkstra's algorithm, Bellman–Ford algorithm, A* search algorithm and Floyd–Warshall algorithm. (need reference)\\
However, this kind of approach has many limitations and disadvantages, for example, programmer must design a corresponding independent algorithm for each kind of path problem. 
And even if the path problem has minor changes to the problem, it is difficult for us to solve the new problem by slightly modifying existing algorithms. 
Besides, when we are solving some complex path problems, such as considering the capacity of the path problem (actually the lexical product of the two set of problems) in the case of giving priority to the distance, Dijkstra algorithm has not guarantee to find all optimal solutions (Although you can find at least one optimal solution) (here need reference, I will find later, and then delete the parentheses).
Hence, lots of predecessors have found the algebraic approaches to work around this kind of problem. 
A simple path problem can be represented by a semiring $(S,\oplus,\otimes,\bar{0},\bar{1})$\cite{carre_algebra_1971,WONGSEELASHOTE197955,dynerowicz_forwarding_2013,mohri_semiring_2002,gurney_lexicographic_2007}. The popular "shortest path problem" can be represented as $(S, min,+,\infty,0)$\cite{mohri_semiring_2002} and the "maximal capacity path problem" can be represented as $(S, max,min, 0, \infty)$. 
There are also an approach that uses the routing matrix to compute the path problem based on the algebraic definition.
The advantages of using the algebraic approaches (the generic algorithms) to the problems is, because of the properties of the operators (such as distributive, commutative) and the multiplication of matrix, one can ignore the detail implementation of the concrete algorithms (Dijkstra's algorithm or Bellman-Ford algorithm for shortest path)\cite{dynerowicz_forwarding_2013}. \\

\section{Motivation}
It is worth mentioning that, the algebraic approaches that using the matrix as the computation rely on the property of the operators a lot, for example, the left and right distributivity of the semiring.
However, most of the cases mentioned above are aimed at some simple problems, or the ideal situations. 
In reality, we need to face the problem that, in the most time for the complex path problem. 
The CAS system can derive most of the properties for the new semirings from the original "simple" simirings. 
However, sometimes the problem set is not the original problem set the provided to us, but the subset of it. For example, as the problem that consist of the lexical product of the shortest path problem semiring and the maximum capacity semiring, there exists some path that have 0 capacity which shouldn't be concerned in the solution of the problem. This kind of refinement is called reduction in our paper, which can be represent as a unary operation on the original problem set.\\
Another example, when we are doing path problems, for the most time the path that has negative value or have infinite value is not quite interesting. However, the operation we defined there is on the whole families of object. When we are defining some data structure, like semiring, we want to know that the properties (like commutative, selective and etc..) of the proper subset (the set of objects after reduction) will hold or not, or for some cases we can't do further calculation.\\
As the method mentioned previously, the algebraic approach to the path problem is depend on the properties of semirings. If one, or some of the properties don't hold for the semiring, the algebraic approach can't guarantee to find the optimal solution (depends on the semiring structure and the operations). Thus, after we applying those reductions on the original problem set, there is no guarantee that the original properties will still hold for the new subset of problem set, and it comes to our point. The existing CAS system doesn't have the functionality to derive and prove the properties for the reduced problem. Hence, I want to figure out the relationship between the reduction and those properties for those operators on the problem set.\\
Furthermore, in most programming languages, it is extremely difficult to express the reduction properly, on contrast, we can represent the reduction as a proof or proposition in our proof world. The second goal to the project is to figure out the friendly-extraction to those reduction.

\section{Aim of the project}
The project has been divided into several steps.\\
Initially I focus on the properties of the reduction operator. Using the properties and the proofs defined and proved in Coq, I defined several combinators that can construct reduction (reduced semigroup) from the existing one.\\
Then, by using those combinators and examples, I can find the properties between the operators and the reduction operation.\\
Then, I should give a friendly-extraction to the outside world.
\section{Outline of the paper}


\chapter{Background} 

A more extensive coverage of what's required to understand your 
work. In general you should assume the reader has a good undergraduate 
degree in computer science, but is not necessarily an expert in 
the particular area you've been working on. Hence this chapter 
may need to summarize some ``text book'' material. 

This is not something you'd normally require in an academic paper, 
and it may not be appropriate for your particular circumstances. 
Indeed, in some cases it's possible to cover all of the ``background'' 
material either in the introduction or at appropriate places in 
the rest of the dissertation. 

\section{Basic Definition}
In the world of logic, we need to define several basic concepts before we are getting started.\\
In mathematics, A binary relation $R$ in an arbitrary sets (or classes) $S$ (here I restrict the relationship to be inside a single set, or say the element from the same type) is a collection of ordered pairs of elements of set (type) $S$, which is a subset of the Cartesian product $S \times S$. In order to link the mathematical concept with the proposition in logic, I provides each relation a representation (hold or not) as a boolean value, which is regarded as a property in $Coq$. $\Lambda S: Type. brel(\text{Binary relation}) : S \rightarrow S \rightarrow bool$. \\
Then, I define the basic operations for a given (but arbitrary) type: binary operation $: \Lambda S.S \rightarrow S \rightarrow S$ and unary operation $: \Lambda S.S \rightarrow S$. \\
And I extend the relation with a product type and a reduced version. I define the composition, identity and production for a unary operation, production and reduction for binary operation. 

\section{Semiring and its Properties}
In abstract algebra, a semiring is a data structure $(S,\oplus,\otimes,\bar0,\bar1)$ where $S$ is a set and $\oplus,\otimes$ are two operators $:S\times S \rightarrow S$. $(S,\oplus)$ is a commutative semigroup (has associative property) and $(S,\otimes)$ is a semigroup: $\forall a,b,c \in S, a \oplus (b\oplus c) = (a \oplus b) \oplus c$, $a \oplus b = b \oplus a$ and $a \otimes (b\otimes c) = (a \otimes b) \otimes c$. $\oplus,\otimes$ are also left and right distributive on $S$ : $\forall a,b,c \in S: a \otimes(b \oplus c) = (a \otimes b) \oplus (a \otimes c)$ and $(a \oplus b) \otimes c  = (a \otimes c) \oplus (b \otimes c)$. $\bar0$ is the identity of $\oplus$ and  $\bar1$ is the identity of $\otimes$: $\forall a \in S, a \oplus \bar{0} = a = \bar{0} \oplus a$ and $\forall a \in S, a \otimes \bar{1} = a = \bar{1} \otimes a$. Finally, $\bar0$ is the  annihilator of $\otimes$: $\forall a \in S, a \otimes \bar{0} = \bar{0} = \bar{0} \otimes a$.

\subsubsection{For Binary Relationship}
Here we have three properties for a binary relationship in set $S$. As it is mentioned in the previous section, in order to define the proposition in the logic proof, the binary relationship will return a boolean value to indicate the relationship is holding or not.\\
Then for a given relationship $\backsim : S \rightarrow S \rightarrow bool$, we have reflexivity: $\forall a \in S, a \backsim a \equiv true$, symmetric : $\forall a, b \in S, a \backsim b \equiv b \backsim a$, and transitivity: $\forall a,b,c \in S, a \backsim b \equiv true \wedge b \backsim c \equiv true \rightarrow a \backsim c \equiv true$.
\subsubsection{For Binary Operation}
Then we have six properties for a binary operator in set $S$. For a given operator $\bullet: S\times S \rightarrow S$ and the equality (actually an arbitrary binary relationship) in $S$, we have:\\
congruence: $\forall s_1,s_2,t_1,t_2 \in S, s_1 \equiv t_1 \wedge s_2 \equiv t_2 \rightarrow s_1 \bullet s_2 \equiv t_1 \bullet t_2$\\
associativity: $\forall a,b,c \in S, a \bullet (b \bullet c) \equiv (a \bullet b) \bullet c$\\
commutativity: $\forall a,b\in S, a \bullet b = b \bullet a$\\
selectivity: $\forall a,b\in S, a \bullet b \in \{a,b\}$\\
hasId: $\exists \bar0 \in S, \forall a \in S, a \bullet \bar0 = a = \bar0 \bullet a$\\
hasAnn: $\exists \bar1 \in S, \forall a \in S, a \bullet \bar1 = \bar1 = \bar1 \bullet a$\\
In this paper, we will mostly concentrate on those properties that are holding or not in the reduced set of problem. 

\section{Reduction}
Algebraic reduction, introduced by Ahnont Wongseelashote in 1977\cite{WONGSEELASHOTE197955} is an unary operator for a given set of problem, $reduce : S \longrightarrow S$. It has several properties, satisfying $reduce(\emptyset) = \emptyset$, $\forall A \in S, reduce(reduce(A)) = reduce(A)$ (which is called idempotent property) and $\oplus : S \times S \rightarrow S, \forall A,B \in S, reduce(reduce(A)\oplus B) = reduce(A\oplus B) = reduce(A\oplus reduce(B))$ (which is left and right invariant property) and this paper will discuss these properties in the later section.\\ 
It is hard to specify the reduced problem set in the most programming language. However, in the world of logic and those programming language that can be used to prove properties, programmer can represent the reduced problem set as $\{x | x \in S, reduce(x) = x\}$ which is also a subset of the original problem set. The idea to represent the reduced set explicitly is to form a pair $<x,Pr(x)>$ where $x \in S$ is the element in the set and $Pr(x) : r(x) = x$ is the proof that the element is in the subset (the element will not change after the reduced function, otherwise it will be reduced). \\
The first example of the reduction is $id$ which maps all the stuff to itself. Another example is the min-set where $min_\leq(x) = \{x \in S | \forall y \in S, \neg(y < x)\}$. Regarding to $\mathcal{P}(S)$ it works well with $\cup$ that the min-set contain all the elements and remove the element that is non-trivial set. Wongseelashote defined the reduction operator in his paper, however the definition is not constructive. Our purpose is defining the reduction constructive in $Coq$, and proving the related properties of the reduction operation.

\section{Algebraic approach to the Path Problem}
As we mentioned above, there are several properties which are really significant to the path problem by using the algebraic approaches. One of them is the distributivity properties for the two operators in a given semiring. 
Left distributivity: $a \otimes (b \oplus c) = (a \otimes b) \oplus (a \otimes c)$ and right distributivity:  $(a \oplus b) \otimes c = (a \otimes c) \oplus (b \otimes c)$. The importance of these two properties is, when we are using matrix multiplication to compute the path problem represent as a semiring, we need there two distributivity properties to guarantee the correctness of the computation (those equation will be shown later). Distributivity is part of the properties inside the semiring, and that becomes one of the reason that semiring has been choosen to represent a path problem.\\
As a result, we modified the original path problem algorithm and the original metric to our new modified metric and matrix equations which is called the generic algorithm.\\
Another important crucial point in our project is "order". As the example provided above, the min-set reduction can be applied in any problem set with some "order". (carnonically order and natural order)\\

\section{Combinator for Algebraic System}
Combinator for Algebraic system (CAS)\cite{griffin_metarouting_2005} is introduced in $L11$. It is a language to design algebraic systems, in which many algebraic properties are automatically received and people can combine different operators to obtain a new semiring\cite{griffin_metarouting_2005}. We can also generalize a more complex path problem (in other words, we can abstract a more complex path problem with this new semiring, such as the lexicographic products \cite{gurney_lexicographic_2007}). CAS can easily return the properties of those combined-operation semirings and it is already defined in Coq\cite{Coq} (mentioned in $L11$ by Dr Timothy Griffin).

\chapter{Related Work} 

This chapter covers relevant (and typically, recent) research 
which you build upon (or improve upon). There are two complementary 
goals for this chapter: 
\begin{enumerate} 
  \item to show that you know and understand the state of the art; and 
  \item to put your work in context
\end{enumerate} 

Ideally you can tackle both together by providing a critique of
related work, and describing what is insufficient (and how you do
better!)

The related work chapter should usually come either near the front or
near the back of the dissertation. The advantage of the former is that
you get to build the argument for why your work is important before
presenting your solution(s) in later chapters; the advantage of the
latter is that don't have to forward reference to your solution too
much. The correct choice will depend on what you're writing up, and
your own personal preference.
\\\\\
Maybe I should put this part into the background part
\section{Algebraic Apporach to the Path Problem}
\section{CAS}
\section{Semirings and Path Spaces}


\chapter{Design and Implementation} 

This chapter may be called something else\ldots but in general 
the idea is that you have one (or a few) ``meat'' chapters which
describe the work you did in technical detail. 
\section{Product Theory and Product Semigroup}
\section{Reduction Properties and Reduction Theory}
Here I define the reduction in a binary operation in three different version: reduce the result of the operation, only reduce the argument of the operation and reduce both result and argument of the operation. 
$bop-reduce : \forall S:Type. \forall r: unary-op S. \forall b: binary-op S.\lambda x,y:S. r(b x y)$, 
$bop-reduce-args : \forall S:Type. \forall r: unary-op S. \forall b: binary-op S.\lambda x,y:S. b (r x) (r y)$, 
$bop-full-reduce : \forall S:Type. \forall r: unary-op S. \forall b: binary-op S.\lambda x,y:S. r(b (r x) (r y))$.

\section{Direct representation of Reduced Semigroup}
(Well defined, but not extraction friendly)\
\subsection{Homomorphism}
\section{Another Representation of Reduced Semigroup -- RSemigroup}
(Given the properties of semigroup together with the reduction proof together.)\\\\
(For those two sections, I am just thinking how to write it properly)
\section{One approach based on the Annihilator}
\chapter{Evaluation} 

For any practical projects, you should almost certainly have
some kind of evaluation, and it's often useful to separate 
this out into its own chapter. \\\\\\


\chapter{Summary and Conclusions} 

As you might imagine: summarizes the dissertation, and draws 
any conclusions. Depending on the length of your work, and 
how well you write, you may not need a summary here. 

You will generally want to draw some conclusions, and point
to potential future work. 




\appendix
\singlespacing

\bibliographystyle{unsrt} 
\bibliography{MphilProject}

\end{document}
